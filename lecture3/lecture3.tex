\documentclass[12pt, twosided]{article}
\input{sdpreamble}
\graphicspath{{./img/}}

\begin{document}
\noindent \textbf{Math 245} \hfill \textbf{Genivive Walsh} \\
\textbf{Kyle Dituro} \hfill \textbf{\today}\hrule
\vspace{.2in}

We examine the free group via the set that it's defined over. In the language of categories \(\mathscr{F}^A\) is the free categoy of \(A\), where the objects are like \(: A \to G\), and the morphisms are group homomorphisms \(\sigma\) like

\begin{center}
  \begin{tikzcd}
    A \arrow[r, "j_1"] \arrow[dr, "j_2"]& G \arrow[d, "\sigma"] \\
    & H
  \end{tikzcd}
\end{center}

Now we define
\begin{df}
  \(F(A)\) is an initial object in \(\mathscr{F}^A\).
\end{df}

\begin{clm}
  The maps \(\opn a \cls \to \langle a \rangle\) are initial in the category.
\end{clm}

This defined \(F(A)\) up to isomorphism, but why do they exist? In particular, define the resolution of the first cancelation relation among the words by \(r: W(A) \to W(A)\), and define moreover \(R:W(A) \to W(A)\) by \(w \longmapsto{R} r^{\lfloor n/2 \rfloor}(w)\). Then \(F(A) = \left(R\left(W\left(A\right)\right), \ \cdot \right)\), where \(\cdot\) denotes concatonation.

\begin{fact}
  This is a group, and this is easy to check for yourself.
\end{fact}
\end{document}
%%% Local Variables:
%%% mode: latex
%%% TeX-master: t
%%% End:

%%% Local Variables:
%%% mode: latex
%%% TeX-master: t
%%% End:
