\documentclass[12pt, twosided]{article}
\usepackage[letterpaper,bindingoffset=0in,%
            left=1in,right=1in,top=1in,bottom=1in,%
            footskip=.25in]{geometry}

\usepackage{mathtools}
\usepackage{graphicx}

%\usepackage{setspace}
%\setstretch{1.1}

\usepackage{amsmath}
\usepackage{amsfonts}
\usepackage{amsthm}
\usepackage{amssymb}
\usepackage{csquotes}
\usepackage{relsize}

\usepackage{tikz}
\usetikzlibrary{cd}
\usetikzlibrary{fit,shapes.geometric}
\tikzset{%  
    mdot/.style={draw, circle, fill=black},
    mset/.style={draw, ellipse, very thick},
}

\usepackage{hhline}
\usepackage{systeme}
\usepackage{mathrsfs}
\usepackage{hyperref}
\usepackage{mathtools}  
\usepackage{silence}
\usepackage{blkarray}
\usepackage{float}
% \usepackage{framed}
\usepackage{mdframed}
\usepackage{array}
\usepackage{stmaryrd}
\usepackage{extarrows}
\usepackage{caption}
\captionsetup[figure]{labelfont={bf},name={Fig.},labelsep=period}

\makeatletter
\newtheoremstyle{indentedDf}
  {8pt}% space before
  {8pt}% space after
  {\addtolength{\@totalleftmargin}{3.5em}
   \addtolength{\linewidth}{-7em}
   \parshape 1 3.5em \linewidth}% body font
  {}% indent
  {\bfseries}% header font
  {.}% punctuation
  {.5em}% after theorem header
  {}% header specification (empty for default)
\makeatother

\makeatletter
\newtheoremstyle{indentedPlain}
  {8pt}% space before
  {8pt}% space after
  {\itshape
   \addtolength{\@totalleftmargin}{3.5em}
   \addtolength{\linewidth}{-7em}
   \parshape 1 3.5em \linewidth}% body font
  {}% indent
  {\bfseries}% header font
  {.}% punctuation
  {.5em}% after theorem header
  {}% header specification (empty for default)
\makeatother

\theoremstyle{indentedDf}
\newtheorem{df}{Definition}[section]
\newtheorem{exa}[df]{Example}
\newtheorem{ques}[df]{Question}
\newtheorem{exr}[df]{Exercise}
\newtheorem{prb}[df]{Problem}
\newtheorem*{notn}{Notation}
\newtheorem*{note}{Note}

\theoremstyle{indentedPlain}
\newtheorem{thm}[df]{Theorem}
\newtheorem{prop}[df]{Proposition}
\newtheorem{conj}[df]{Conjecture}
\newtheorem{cor}[df]{Corollary}
\newtheorem{lm}[df]{Lemma}
\newtheorem*{fact}{Fact}
\newtheorem*{idea}{Idea}
\newtheorem*{clm}{Claimn}
\newtheorem*{rmk}{Remark}
\usepackage[ruled]{algorithm2e}

\usepackage{ulem}
\makeatletter

\def\lf{\left\lfloor}   
\def\rf{\right\rfloor}
\def\lc{\left\lceil}   
\def\rc{\right\rceil}
\def\st{\text{ s.t. }}
\def\ker{\mathrm{ker}\ }
\def\im{{\mathrm{im}}\ }
\def\1{^{-1}}
\def\2{^2}
\def\3{^3}
\def\tn{^n}
\def\ind{\mathbf{1}}
\def\R{\mathbb{R}}
\def\Q{\mathbb{Q}}
\def\Z{\mathbb{Z}}
\def\C{\mathbb{C}}
\def\I{\mathbb{I}}
\def\N{\mathbb{N}}
\def\F{\mathbb{F}}
\def\A{\mathbb{A}}
\def\Li{\text{Li}}
\def\th{^\text{th}}
\def\sp{\text{Sp}}
\def\opn{\left\{}
\def\cls{\right\}}
\def\Aut{\text{Aut}}
\def\PG{\text{PG}}
\def\GL{\text{GL}}
\def\PGL{\text{PGL}}
\def\Cov{\text{Cov}}
\def\Pack{\text{Pack}}
\def\PgamL{\text{P}\Gamma\text{L}}
\def\gamL{\Gamma\text{L}}
\def\cl{\text{cl}}
\def\stbar{\ \middle\vert\ }
\def\partdone{\hphantom{1} \hfill \(\triangle\)}
\def\s0{_0}
\def\s1{_1}
\def\s2{_2}
\def\id{\mathrm{id}}
\def\topn{\text{ open}}
\def\Bd{\text{Bd }}
\def\nope{\(\longrightarrow\!\!\longleftarrow\)}
\def\stt{\(^{\text{st}}\)}
\def\tht{\(^{\text{th}}\)}
\def\ndt{\(^{\text{nd}}\)}
\def\t{^{T}}
\def\c{^c}
\newcommand\norm[1]{\left\lVert#1\right\rVert}
\renewcommand{\P}{\mathbb{P}}
\newcommand{\leg}[2]{\left( \frac{#1}{#2} \right)}

\renewcommand*\env@matrix[1][*\c@MaxMatrixCols c]{%
   \hskip -\arraycolsep
   \let\@ifnextchar\new@ifnextchar
   \array{#1}}
\makeatother

% These two lines suppress the warning generated 
% by amsmath for overwriting the choose command  
% because it's annoying. This probably has unint-
% ended ramifications somewhere else, but I'm too
% lazy to actually figure that out, so we'll cro-
% ss that bridge when we come to it lol.
\renewcommand{\choose}[2]{\left( {#1 \atop #2} \right)}
\WarningFilter{amsmath}{Foreign command} 

\renewcommand{\mod}[1]{\ (\mathrm{mod}\ #1)}
\renewcommand{\vec}[1]{\mathbf{#1}}

\let\oldprime\prime
\def\prime{^\oldprime}

\usepackage{float}
\restylefloat{figure}

\usepackage{cleveref}
\Crefname{thm}{Theorem}{Theorems}

% Comment commands for co-authors
\newcommand{\kmd}[1]{{\color{purple} #1}}

\newcolumntype{L}{>{$}l<{$}}
% Bib matter
\let\oldepsilon\epsilon
\def\epsilon{\varepsilon}

\let\oldphi\phi
\def\phi{\varphi}

% New definition of square root:
% it renames \sqrt as \oldsqrt
\let\oldsqrt\sqrt
% it defines the new \sqrt in terms of the old one
\def\sqrt{\mathpalette\DHLhksqrt}
\def\DHLhksqrt#1#2{%
\setbox0=\hbox{$#1\oldsqrt{#2\,}$}\dimen0=\ht0
\advance\dimen0-0.2\ht0
\setbox2=\hbox{\vrule height\ht0 depth -\dimen0}%
{\box0\lower0.4pt\box2}}
%%% Local Variables:
%%% mode: plain-tex
%%% TeX-master: t
%%% End:

\graphicspath{{./img/}}

\begin{document}
\noindent \textbf{Math 245} \hfill \textbf{Genevieve Walsh } \\
\textbf{Kyle Dituro} \hfill \textbf{\today}\hrule
\vspace{.2in}

We examine the free group via the set that it's defined over. In the language of categories \(\mathscr{F}^A\) is the free categoy of \(A\), where the objects are like \(: A \to G\), and the morphisms are group homomorphisms \(\sigma\) like

\begin{center}
  \begin{tikzcd}
    A \arrow[r, "j_1"] \arrow[dr, "j_2"]& G \arrow[d, "\sigma"] \\
    & H
  \end{tikzcd}
\end{center}

Now we define
\begin{df}
  \(F(A)\) is an initial object in \(\mathscr{F}^A\).
\end{df}

\begin{clm}
  The maps \(\opn a \cls \to \langle a \rangle\) are initial in the category.
\end{clm}

This defined \(F(A)\) up to isomorphism, but why do they exist? In particular, define the resolution of the first cancelation relation among the words by \(r: W(A) \to W(A)\), and define moreover \(R:W(A) \to W(A)\) by \(w \longmapsto{R} r^{\lfloor n/2 \rfloor}(w)\). Then \(F(A) = \left(R\left(W\left(A\right)\right), \ \cdot \right)\), where \(\cdot\) denotes concatonation.

\begin{fact}
  This is a group, and this is easy to check for yourself.
\end{fact}

Then define the map \(j:A \to F(A)\) as the map that takes a letter as a set object to a letter as a word in the group \(F(A)\). Then the homomorphisms \(\sigma\) are defined letterwise in order to force the homomorphism condition.

In other words, just take \(abc \xmapsto{\sigma} j(a)j(b)j(c)\).

\section{Subgroups}

\begin{df}
  A \textit{Subgroup} \(H\) of a group \(G\) (denoted by \(H \leq G\)) is a subset \(H \subseteq G\) such that \(H\) is a group inheriting the group operation of \(G\).
\end{df}

\begin{lm}
  \(H \subseteq G\) is a subgroup iff \(\forall a, b \in H,\ ab\1 \in H\).
\end{lm}
\begin{proof}
  Trivially verified.
\end{proof}

\begin{exa}
  The image of a group homomorphism \(\phi\) is a subgroup of the domain.
\end{exa}

Now we can define a slightly more interesting object.

\begin{df}
  A subgroup \(H \leq G\) is called \textrm{normal} if \(g\in G\) \(gHg\1 \in H\)
\end{df}

\begin{exa}
  \(\ker \phi\) is a normal subgroup of the domain of \(\phi\) since \(\forall h \in H, g \in G\), \(\phi(g)\phi(h)\phi(g)\1 = \phi(g)\phi(g)\1 = e\)
\end{exa}

\begin{clm}
  All normal subgroups are the kernel of some homomorphism.
\end{clm}
\begin{proof}
  First notice that the left cosets of some normal subgroup \(K \trianglelefteq G\) partition \(G\). Now we claim that this gives us a well-defined group operation. This can be seen by noticing that

  \begin{align*}
    gK \cdot hK &= ghK \\
    gk_gK \cdot hk_hK &= gk_ghk_hK\\
                &= ghk_g\prime k_hK \\
                &=ghK.  
  \end{align*}
  when \(K\) is normal, (where \(kh = hk\prime\) for \(k\prime \in K\)).

  Now we denote this group \(G / K\), with \(g \xmapsto{\pi} gK\).
\end{proof}

Then as a corollary of this, we get the following:

\begin{cor}
  If \(\phi: G \to G\prime\) is onto, then
  \begin{align*}
    G / \ker{\phi} \cong G\prime
  \end{align*}
\end{cor}

And moreover,
\begin{thm}[LaGrange's theorem]
  The order of any subgroup divides \(|G|\). in other words:
  \(|G| = [G:H]|H|\).
\end{thm}

Quotient groups allow one to say that if \(\phi: G \to G\prime\):

\begin{center}
  \begin{tikzcd}
    G \arrow[r, "\pi"] \arrow[rr, bend right=25, "\phi"] & G/\ker \phi \cong \textrm{Im} \phi \arrow[r, hookrightarrow]& G\prime
  \end{tikzcd}
\end{center}

\section{Group Actions}

The \textit{action} of a group on a set is a homomorphism \[\sigma: G \to \mathrm{Aut}(A).\] Namely, a left action \(\rho: G \times A \to A\) is defined such that \(\rho(gh, a) = \rho(g, \rho(h, a))\).

\begin{fact}
  Every group acts faithfully on some set. Therefore it is a subgroup of a permutation group. Yes, we are just stating Cayley's theorem as a fact, eat shit idiot.
\end{fact}
\end{document}
%%% Local Variables:
%%% mode: latex
%%% TeX-master: t
%%% End:

%%% Local Variables:
%%% mode: latex
%%% TeX-master: t
%%% End:
