\documentclass[12pt, twosided]{article}
\input{sdpreamble}
\graphicspath{{./img/}}

\begin{document}
\noindent \textbf{Math 245} \hfill \textbf{Genevieve Walsh } \\
\textbf{Kyle Dituro} \hfill \textbf{\today}\hrule
\vspace{.2in}

We examine the free group via the set that it's defined over. In the language of categories \(\mathscr{F}^A\) is the free categoy of \(A\), where the objects are like \(: A \to G\), and the morphisms are group homomorphisms \(\sigma\) like

\begin{center}
  \begin{tikzcd}
    A \arrow[r, "j_1"] \arrow[dr, "j_2"]& G \arrow[d, "\sigma"] \\
    & H
  \end{tikzcd}
\end{center}

Now we define
\begin{df}
  \(F(A)\) is an initial object in \(\mathscr{F}^A\).
\end{df}

\begin{clm}
  The maps \(\opn a \cls \to \langle a \rangle\) are initial in the category.
\end{clm}

This defined \(F(A)\) up to isomorphism, but why do they exist? In particular, define the resolution of the first cancelation relation among the words by \(r: W(A) \to W(A)\), and define moreover \(R:W(A) \to W(A)\) by \(w \longmapsto{R} r^{\lfloor n/2 \rfloor}(w)\). Then \(F(A) = \left(R\left(W\left(A\right)\right), \ \cdot \right)\), where \(\cdot\) denotes concatonation.

\begin{fact}
  This is a group, and this is easy to check for yourself.
\end{fact}

Then define the map \(j:A \to F(A)\) as the map that takes a letter as a set object to a letter as a word in the group \(F(A)\). Then the homomorphisms \(\sigma\) are defined letterwise in order to force the homomorphism condition.

In other words, just take \(abc \xmapsto{\sigma} j(a)j(b)j(c)\).

\section{Subgroups}

\begin{df}
  A \textit{Subgroup} \(H\) of a group \(G\) (denoted by \(H \leq G\)) is a subset \(H \subseteq G\) such that \(H\) is a group inheriting the group operation of \(G\).
\end{df}

\begin{lm}
  \(H \subseteq G\) is a subgroup iff \(\forall a, b \in H,\ ab\1 \in H\).
\end{lm}
\begin{proof}
  Trivially verified.
\end{proof}

\begin{exa}
  The image of a group homomorphism \(\phi\) is a subgroup of the domain.
\end{exa}

Now we can define a slightly more interesting object.

\begin{df}
  A subgroup \(H \leq G\) is called \textrm{normal} if \(g\in G\) \(gHg\1 \in H\)
\end{df}

\begin{exa}
  \(\ker \phi\) is a normal subgroup of the domain of \(\phi\) since \(\forall h \in H, g \in G\), \(\phi(g)\phi(h)\phi(g)\1 = \phi(g)\phi(g)\1 = e\)
\end{exa}

\begin{clm}
  All normal subgroups are the kernel of some homomorphism.
\end{clm}
\begin{proof}
  First notice that the left cosets of some normal subgroup \(K \trianglelefteq G\) partition \(G\). Now we claim that this gives us a well-defined group operation. This can be seen by noticing that

  \begin{align*}
    gK \cdot hK &= ghK \\
    gk_gK \cdot hk_hK &= gk_ghk_hK\\
                &= ghk_g\prime k_hK \\
                &=ghK.  
  \end{align*}
  when \(K\) is normal, (where \(kh = hk\prime\) for \(k\prime \in K\)).

  Now we denote this group \(G / K\), with \(g \xmapsto{\pi} gK\).
\end{proof}

Then as a corollary of this, we get the following:

\begin{cor}
  If \(\phi: G \to G\prime\) is onto, then
  \begin{align*}
    G / \ker{\phi} \cong G\prime
  \end{align*}
\end{cor}

And moreover,
\begin{thm}[LaGrange's theorem]
  The order of any subgroup divides \(|G|\). in other words:
  \(|G| = [G:H]|H|\).
\end{thm}

Quotient groups allow one to say that if \(\phi: G \to G\prime\):

\begin{center}
  \begin{tikzcd}
    G \arrow[r, "\pi"] \arrow[rr, bend right=25, "\phi"] & G/\ker \phi \cong \textrm{Im} \phi \arrow[r, hookrightarrow]& G\prime
  \end{tikzcd}
\end{center}

\section{Group Actions}

The \textit{action} of a group on a set is a homomorphism \[\sigma: G \to \mathrm{Aut}(A).\] Namely, a left action \(\rho: G \times A \to A\) is defined such that \(\rho(gh, a) = \rho(g, \rho(h, a))\).

\begin{fact}
  Every group acts faithfully on some set. Therefore it is a subgroup of a permutation group. Yes, we are just stating Cayley's theorem as a fact, eat shit idiot.
\end{fact}
\end{document}
%%% Local Variables:
%%% mode: latex
%%% TeX-master: t
%%% End:

%%% Local Variables:
%%% mode: latex
%%% TeX-master: t
%%% End:
