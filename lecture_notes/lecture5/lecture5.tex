\documentclass[12pt, twosided]{article}
\input{sdpreamble}
\graphicspath{{./img/}}

\begin{document}
\noindent \textbf{Math 245} \hfill \textbf{Genevieve Walsh} \\
\textbf{Kyle Dituro} \hfill \textbf{January 31\stt, 2023}\hrule
\vspace{.2in}

\begin{center}
  {\huge \textbf{These notes feel really bad. Use at your own risk}}
\end{center}
\vspace{.2in}
\section{Products}


Suppose that you have \(H, K < G\), and consider the product \(H \times K\).

\begin{ques}
  When is \(HK \cong H \times K\) (where \(HK\) are the \(h\) cosets of \(K\)).
\end{ques}

Well we know that \(HK\) is a subgroup precisely when \(H\) or \(K\) is a subgroup. And then with \(h_2k_2\),
\begin{align*}
  h_1k_1k_2\1h_2\1 \in HK
\end{align*}.

Then if \(K\) in particulare is normal, then
\begin{align*}
  h_1k_1k_2\1 h_2\1 &= h_1h_2\1 h_2k_1k_2\1 h_2 \1 \\
                   &=h_1h_2\1 k_3
\end{align*}

And if \(H\) is normal:

\begin{align*}
  h_1(k_1k-2\1h_2\1) = = h\1h_3k_1k_2\1.
\end{align*}

because \(khk\1 = h\1\).

If \(H\) or \(k\) is normal, when is
\begin{align*}
  HK \cong H \times K?
\end{align*}

Since these things are both normal, we'll rename them to \(N\) and \(K\). \([N, K] \cong e\), then everything in \(N\) commutes with everything in \(K\). If \(N\) and \(K\) are both normal in \(G\), then \([N, K] \subseteq N \cap K\).

So take a generator \((n_1k_1n_1\1)k_1\1 \in N\).

\begin{thm}
  If \(N, K \trianglelefteq G\)  and \(N \cap K = e\), then \(NK \cong N \times K\).
\end{thm}
\begin{proof}
  \begin{align*}
    \phi: N \times K \to NK \\
    (n,k) \mapsto nk.
  \end{align*}

  \(\phi(n_1n_2, k_1k_2) = n_1n_2k_1k_2\) so \(\phi(n_1, k_1) \cdot \phi(n_2, k_2)\) which equals \(n_1n_2k_1k_2\) because everything in \(N\) commutes with everything in \(K\), so we have a bijection.
\end{proof}

\begin{lm}
  If \(|G| = pq\ p <q\), are prime, and \(G\) contains normal subgroups \(H, |H| = p\), \(K, |K| = q\), then \(G\) is cyclic.
\end{lm}
\begin{proof}
  \(H \cap K\) is a subgroup of \(H\) and \(K\). \(|H \cap K| \mid |H|\) and \(|H \cap k| \mid |k|\) so \(|H \cap K| = 1\).

  Thus \(HK \cong H \times K\), and moreover \(HK = G\) by the counting argument since \(|HK| = |G|\). Thus \(G \cong H \times K \cong \Z_p \times \Z_q\) generated by \((1,1)\).
\end{proof}

Now what if only one of them is normal?

\begin{exa}
  Consider \(S_3 \cong D_6\) with presentaiton \(\left\langle s, r \stbar s^2 = 1, r^3 = 1,\ sr = r\1s \right\rangle\).

  Then \(\langle r^3 \rangle\) is normal (since it is of index 2 in \(D_6\)). Then notice that the product \(\langle s \rangle \times \langle r \rangle\) is not isomorphic to \(D_6\). 
\end{exa}

Suppose that we have \(G = NH\) with \(N \cap H = e\), and \(N \trianglerighteq N\). then \(G/N\) is a group isomorphic to \(H\). Namely, the sequence

\begin{center}
  \begin{tikzcd}
    1 \rar & N \rar{\phi} & G \rar{\psi} & H  \rar& 1
  \end{tikzcd}
\end{center}

Remember: an exact sequence splits if \(G\) contains a copy of \(H\) where \(N \cap H = e\), and

\begin{center}
  \begin{tikzcd}
    1 \rar & N \rar{} & G \rar{} \rar[leftarrow, bend left]{s} & H  \rar  & 1.
  \end{tikzcd}
\end{center}

\begin{exa}
  In \(D_6\), we have
  \begin{center}
    \begin{tikzcd}
          1 \rar & Z_3 \rar{} & D_6 \rar{} \rar[leftarrow, bend left]{s} & \Z_2  \rar  & 1.
    \end{tikzcd}
  \end{center}
  splits.
\end{exa}

\begin{exa} Using \(\Z\):
  
  \begin{center}
    \begin{tikzcd}
          1 \rar & \Z \rar{a \mapsto a^8} & \Z \rar{} \rar[leftarrow, bend left] & \Z /8\Z  \rar  & 1.
    \end{tikzcd}
  \end{center}
  \textbf{doesn't} split.
\end{exa}

now from the counting equation:

\(G \actson S\), \[|S| = |Z| = \sum_{a \in A}^n |O_a|\] where \(A\) contains 1 element from each non-trivial orbit.

\begin{df}
  For some prime \(p\), a \(p\)-group is a group of order \(p^n\).
\end{df}

Now suppose we have a \(p\)-group with \(G \actson S\). so \(|Z| = |S| \mod{p}\)

Now we can provide a proof of Cauchy's theorem from earlier.

\begin{proof}
  \(|G|< p\), \(p\) a prime such that \(p \divides |G|\). Then \(G\) contains an element of order \(p\).

  Consider the set \(S\) of \(p\)-tuples in \(G\) which multiply to \(e\) via the grop operation.

  Then \(\Z_p\) acts on \(S\) by cyclic permutations. Then the cyclic permutations act on the set, and the fixed points are elements of order \(p\) or \(1\).

  But then \(|Z| \equiv |S| \equiv 0 \mod{p}\), and it is not zero since there exists an element of order \(p\). So the number of non trivial subgroups is \(1 \mod{p}\).
\end{proof}

\begin{clm}[Cauchy +]
  If \(G\) is a finite group, and \(p \divides |G|\), where \(p\) is prime, \(N = \) number of cyclic groups with size \(p\), then \(N \equiv 1 \mod{p}\)
\end{clm}

\begin{exa}
  If \(p > 0\) is prime, and \(m > 0\), if \(|G| = mp\) with \(m < p\), then \(G\) is not simple.
\end{exa}

Then also the number of subgroups with \(p\) elements is congruant to \(1\mod{p}\). Actually, if there's only \(1\), then it is normal.

Now we state Sylow's theorems.

\begin{df}
  Let \(|G| = p^rm\), where \((p, m) = 1\), then a \textit{Sylow}-\(p\) \textit{subgroup} is a subgroup of size \(p^r\)
\end{df}

\begin{thm}[Sylow I]
  G contains a subgroup of size \(p^r\).
\end{thm}

\begin{thm}[Sylow II]
  All Sylow \(p\) subgroups of \(G\) are conjugate
\end{thm}

\begin{thm}
  The number of Sylow \(p\)-subgroups is congruant to \(1 \mod{p}\) and divides \(m\).
\end{thm}

\begin{exa}
  There are no simple groups of size \(12\). By (III), there must by \(1\) or \(4\) Sylow 3 subgroups, so suppose it's 4. Then the number of non-trivial elements in the Sylow-3 subgroup of order \(3\) is \(2 \times 4 = 8\). But then there are 4 elements left for the single subgroup of order 4, but this takes up the rest of the grou, and thus it is normal.
\end{exa}

\begin{exr}
  If \(G\) is a group of order \(33\), then \(G\) is cyclic. Prove this fact, and generalize for \(p\) and \(q\).
\end{exr}
\end{document}
%%% Local Variables:
%%% mode: latex
%%% TeX-master: t
%%% End:

%%% Local Variables:
%%% mode: latex
%%% TeX-master: t
%%% End:
