\documentclass[12pt, twosided]{article}
\usepackage[letterpaper,bindingoffset=0in,%
            left=1in,right=1in,top=1in,bottom=1in,%
            footskip=.25in]{geometry}

\usepackage{mathtools}
\usepackage{graphicx}

%\usepackage{setspace}
%\setstretch{1.1}

\usepackage{amsmath}
\usepackage{amsfonts}
\usepackage{amsthm}
\usepackage{amssymb}
\usepackage{csquotes}
\usepackage{relsize}

\usepackage{tikz}
\usetikzlibrary{cd}
\usetikzlibrary{fit,shapes.geometric}
\tikzset{%  
    mdot/.style={draw, circle, fill=black},
    mset/.style={draw, ellipse, very thick},
}

\usepackage{hhline}
\usepackage{systeme}
\usepackage{mathrsfs}
\usepackage{hyperref}
\usepackage{mathtools}  
\usepackage{silence}
\usepackage{blkarray}
\usepackage{float}
% \usepackage{framed}
\usepackage{mdframed}
\usepackage{array}
\usepackage{stmaryrd}
\usepackage{extarrows}
\usepackage{caption}
\captionsetup[figure]{labelfont={bf},name={Fig.},labelsep=period}

\makeatletter
\newtheoremstyle{indentedDf}
  {8pt}% space before
  {8pt}% space after
  {\addtolength{\@totalleftmargin}{3.5em}
   \addtolength{\linewidth}{-7em}
   \parshape 1 3.5em \linewidth}% body font
  {}% indent
  {\bfseries}% header font
  {.}% punctuation
  {.5em}% after theorem header
  {}% header specification (empty for default)
\makeatother

\makeatletter
\newtheoremstyle{indentedPlain}
  {8pt}% space before
  {8pt}% space after
  {\itshape
   \addtolength{\@totalleftmargin}{3.5em}
   \addtolength{\linewidth}{-7em}
   \parshape 1 3.5em \linewidth}% body font
  {}% indent
  {\bfseries}% header font
  {.}% punctuation
  {.5em}% after theorem header
  {}% header specification (empty for default)
\makeatother

\theoremstyle{indentedDf}
\newtheorem{df}{Definition}[section]
\newtheorem{exa}[df]{Example}
\newtheorem{ques}[df]{Question}
\newtheorem{exr}[df]{Exercise}
\newtheorem{prb}[df]{Problem}
\newtheorem*{notn}{Notation}
\newtheorem*{note}{Note}

\theoremstyle{indentedPlain}
\newtheorem{thm}[df]{Theorem}
\newtheorem{prop}[df]{Proposition}
\newtheorem{conj}[df]{Conjecture}
\newtheorem{cor}[df]{Corollary}
\newtheorem{lm}[df]{Lemma}
\newtheorem*{fact}{Fact}
\newtheorem*{idea}{Idea}
\newtheorem*{clm}{Claimn}
\newtheorem*{rmk}{Remark}
\usepackage[ruled]{algorithm2e}

\usepackage{ulem}
\makeatletter

\def\lf{\left\lfloor}   
\def\rf{\right\rfloor}
\def\lc{\left\lceil}   
\def\rc{\right\rceil}
\def\st{\text{ s.t. }}
\def\ker{\mathrm{ker}\ }
\def\im{{\mathrm{im}}\ }
\def\1{^{-1}}
\def\2{^2}
\def\3{^3}
\def\tn{^n}
\def\ind{\mathbf{1}}
\def\R{\mathbb{R}}
\def\Q{\mathbb{Q}}
\def\Z{\mathbb{Z}}
\def\C{\mathbb{C}}
\def\I{\mathbb{I}}
\def\N{\mathbb{N}}
\def\F{\mathbb{F}}
\def\A{\mathbb{A}}
\def\Li{\text{Li}}
\def\th{^\text{th}}
\def\sp{\text{Sp}}
\def\opn{\left\{}
\def\cls{\right\}}
\def\Aut{\text{Aut}}
\def\PG{\text{PG}}
\def\GL{\text{GL}}
\def\PGL{\text{PGL}}
\def\Cov{\text{Cov}}
\def\Pack{\text{Pack}}
\def\PgamL{\text{P}\Gamma\text{L}}
\def\gamL{\Gamma\text{L}}
\def\cl{\text{cl}}
\def\stbar{\ \middle\vert\ }
\def\partdone{\hphantom{1} \hfill \(\triangle\)}
\def\s0{_0}
\def\s1{_1}
\def\s2{_2}
\def\id{\mathrm{id}}
\def\topn{\text{ open}}
\def\Bd{\text{Bd }}
\def\nope{\(\longrightarrow\!\!\longleftarrow\)}
\def\stt{\(^{\text{st}}\)}
\def\tht{\(^{\text{th}}\)}
\def\ndt{\(^{\text{nd}}\)}
\def\t{^{T}}
\def\c{^c}
\newcommand\norm[1]{\left\lVert#1\right\rVert}
\renewcommand{\P}{\mathbb{P}}
\newcommand{\leg}[2]{\left( \frac{#1}{#2} \right)}

\renewcommand*\env@matrix[1][*\c@MaxMatrixCols c]{%
   \hskip -\arraycolsep
   \let\@ifnextchar\new@ifnextchar
   \array{#1}}
\makeatother

% These two lines suppress the warning generated 
% by amsmath for overwriting the choose command  
% because it's annoying. This probably has unint-
% ended ramifications somewhere else, but I'm too
% lazy to actually figure that out, so we'll cro-
% ss that bridge when we come to it lol.
\renewcommand{\choose}[2]{\left( {#1 \atop #2} \right)}
\WarningFilter{amsmath}{Foreign command} 

\renewcommand{\mod}[1]{\ (\mathrm{mod}\ #1)}
\renewcommand{\vec}[1]{\mathbf{#1}}

\let\oldprime\prime
\def\prime{^\oldprime}

\usepackage{float}
\restylefloat{figure}

\usepackage{cleveref}
\Crefname{thm}{Theorem}{Theorems}

% Comment commands for co-authors
\newcommand{\kmd}[1]{{\color{purple} #1}}

\newcolumntype{L}{>{$}l<{$}}
% Bib matter
\let\oldepsilon\epsilon
\def\epsilon{\varepsilon}

\let\oldphi\phi
\def\phi{\varphi}

% New definition of square root:
% it renames \sqrt as \oldsqrt
\let\oldsqrt\sqrt
% it defines the new \sqrt in terms of the old one
\def\sqrt{\mathpalette\DHLhksqrt}
\def\DHLhksqrt#1#2{%
\setbox0=\hbox{$#1\oldsqrt{#2\,}$}\dimen0=\ht0
\advance\dimen0-0.2\ht0
\setbox2=\hbox{\vrule height\ht0 depth -\dimen0}%
{\box0\lower0.4pt\box2}}
%%% Local Variables:
%%% mode: plain-tex
%%% TeX-master: t
%%% End:

\graphicspath{{./img/}}

\begin{document}
\noindent \textbf{Math 245} \hfill \textbf{Genevieve Walsh} \\
\textbf{Kyle Dituro} \hfill \textbf{February 7\tht, 2024}\hrule
\vspace{.2in}

\section{Quaternions}

We view the quaternions \(\mathcal{H}\) as a ``complexification'' of \(\R^4\) when given a ring structure. Namely, the elements look like
\begin{align*}
  a + bi + cj + dk, \quad (a,b,c,d \in \R),
\end{align*}
Where we identify \(i^2 = j^2 = k^2 = -1\), and \( ij = k,\ jk = i,\) and \(ki = j\). Notice that from this we get the normal negative relations implied.

We also get a very nice norm in \(\mathcal{H}\), given by quaternion conjugation by inheriting the norm from \(\R^4\) as

\begin{align*}
  \vec v \overline{\vec v} = (a + bi + cj + dj)(a -bi -cj - dk) = a^2 + b^2 + c^2 + d^2.
\end{align*}

Notice also that \(\mathcal{H}\) is a division ring, and that in particular \(\vec v\1 = \frac{\overline{\vec v}}{||\vec v||}\).

\section{Subrings and Ideals}
\begin{note}
  I didn't typeset the extended discussion of polynomial rings. It's mostly a trivial construction in the category \(\mathcal{R}_A\). However, there is the interesting idea of indexing sums to represent polynomials in arbitrarily many variables. It's worth noting for that a polynomial in \(\frac{k[X_1, \ldots X_n]}{\langle X_1^d, \ldots, X_n^d \rangle}\), you will be able to index your sum over the partitions of \(d\), so we can notice that there are precisely \(\choose{n + d}{n + 1}\) terms in the sum. 
\end{note}

As we always do when studying a thing in math, we've studied morphisms between these things, so now we'll talk about substructure.
\begin{df}
  A \textit{subring} of a ring \(R\) is a subset of \(R\) which inherits the ring structure and \(S \hookrightarrow R\) is a homomorphism
\end{df}

\subsection{The endomorphism group}

Let \(G\) be an abelian group, and let \(\mathrm{End}_{Ab}(G)\) be the endomorphisms. Then this is a ring in the obvious way (addition is pointwse, multiplication is composition).

\begin{exa}
  Notice that \(\mathrm{End}_{\mathrm{AB}}(\Z) \cong \Z\) as rings, with the isomorphism given by evaluation at \(1\). 
\end{exa}

\subsection{Cayley's Theorem for Rings}

Cayley's theorem, which states that every group is a subgroup of a symmetric group, in particular states that a group acts on itself by left-multiplication. Now we define

\begin{df}
  Take \(r \in R\), then let \(\lambda_r:R \to R\) be given by \(\lambda_r(s) = rs\).
\end{df}

\begin{thm}
  Let \(R\) be a ring. Then \(r \xmapsto{\lambda} \lambda_r\) is an injective ring homomorphism.
\end{thm}

The proof of this statement is actually just boiling down to checking that the map is indeed an injective homomorphism. Nothing new. We've admitted it.

\subsection{Ideals}
Ideals are the analogue of normal subgroups for rings.

\begin{df}
  An ideal \(I\) of a ring \(R\)  is a subgroup such that
  \begin{align*}
    lI \subseteq I \quad (\forall l \in R) \\
    Ir \subseteq I \quad (\forall r \in R)
  \end{align*}

  Taking only one of these two conditions will give the definition of a left or right ideal respectively.
\end{df}

So now we can study what's so nice about ideals:

\begin{thm}
  Let \(\phi: R \to S\) be a ring homomorphism. Then \(\mathrm{ker} \phi\) is an ideal.
\end{thm}

\subsection{Quotients}

We define a quotient ring in the same way that we define quotient groups on groups, simply replacing our language of normal subgroups with our language of ideals.

\begin{df}
  Let \(I \leq R\) be an ideal. Then \(R / I\) is an abelian group with
  \begin{align*}
    (r + I) + (s + I) = (r + s) + I = (s + r) + I
  \end{align*}

  and a ring with
  \begin{align*}
    (r + I) \cdot (s + I) = r \cdot s + I
  \end{align*}
\end{df}

Showing that this is well defined is no trick.

Suppose that \(r_1 - r_2 \in I,\) and \(s_1 - s_2 \in I\), in other words, \(r_1\) and \(r_2\) are in the same coset of our ideal (and for \(s\)). Then consider \(r_1s_1 - r_2s_2 = r_1(s_1 - s_2) + (r_1 - r_2)s_2\), so we have well definedness.

\begin{exa}
  \(\Z / n\Z\) is a ring.

  Consider the canonical map
  \begin{align*}
    \Z \to R \\
    n \mapsto \underbrace{1_R + \ldots + 1_R}_{n\ \mathrm{times}}
  \end{align*}

  the kernel of this map is an ideal, and is either \(n\Z\) or 0
\end{exa}
From this example, there is a natural definition
\begin{df}
  As in the example above, \(n\) is the \textit{characteristic} of \(R\). If the ideal generated is the 0 ideal, then the ring is characteristic 0.
\end{df}

\begin{thm}
  Let \(I\) be an ideal of \(R\). Then for every ring homomorphism \(\phi: R \to S\) such that \(I \subseteq \mathrm{ker}\phi\), there is a unique ring homomorphism \(\overline{\phi}\) such that the diagram

  \begin{center}
    \begin{tikzcd}
      R \rar{\phi} & S \dar[leftarrow]{\overline{\phi}}\\
      & R/I  \ular[leftarrow]{\pi}
    \end{tikzcd}
  \end{center}
  commutes.
\end{thm}

Then just like for groups, we get that if \(\phi: R \to S\) is a ring homomorphism, then

\begin{center}
  \begin{tikzcd}
    R \arrow[twoheadrightarrow, r] \arrow[rrr, bend right] & R/\textrm{ker}(\phi) \arrow[r, "\cong"] & \lar[leftarrow, "\phi"] \textrm{im} \phi \arrow[r, hookrightarrow] & S
  \end{tikzcd}
\end{center}
and hence the first isomorphism theorem for rings:

\begin{thm}
  Let \(\phi: R \to S\) be a surjective map, then
  \begin{align*}
    R / \mathrm{ker} \phi \cong S
  \end{align*}
\end{thm}
\end{document}
%%% Local Variables:
%%% mode: latex
%%% TeX-master: t
%%% End:

%%% Local Variables:
%%% mode: latex
%%% TeX-master: t
%%% End:



