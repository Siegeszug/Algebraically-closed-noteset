\documentclass[12pt, twosided]{article}
\input{sdpreamble}
\graphicspath{{./img/}}

\begin{document}
\noindent \textbf{Math 245} \hfill \textbf{Genevieve Walsh} \\
\textbf{Kyle Dituro} \hfill \textbf{February 7\tht, 2024}\hrule
\vspace{.2in}

\section{Quaternions}

We view the quaternions \(\mathcal{H}\) as a ``complexification'' of \(\R^4\) when given a ring structure. Namely, the elements look like
\begin{align*}
  a + bi + cj + dk, \quad (a,b,c,d \in \R),
\end{align*}
Where we identify \(i^2 = j^2 = k^2 = -1\), and \( ij = k,\ jk = i,\) and \(ki = j\). Notice that from this we get the normal negative relations implied.

We also get a very nice norm in \(\mathcal{H}\), given by quaternion conjugation by inheriting the norm from \(\R^4\) as

\begin{align*}
  \vec v \overline{\vec v} = (a + bi + cj + dj)(a -bi -cj - dk) = a^2 + b^2 + c^2 + d^2.
\end{align*}

Notice also that \(\mathcal{H}\) is a division ring, and that in particular \(\vec v\1 = \frac{\overline{\vec v}}{||\vec v||}\).

\section{Subrings and Ideals}
\begin{note}
  I didn't typeset the extended discussion of polynomial rings. It's mostly a trivial construction in the category \(\mathcal{R}_A\). However, there is the interesting idea of indexing sums to represent polynomials in arbitrarily many variables. It's worth noting for that a polynomial in \(\frac{k[X_1, \ldots X_n]}{\langle X_1^d, \ldots, X_n^d \rangle}\), you will be able to index your sum over the partitions of \(d\), so we can notice that there are precisely \(\choose{d-1}{n-1}\) terms in the sum. 
\end{note}

As we always do when studying a thing in math, we've studied morphisms betewen these things, so now we'll talk about substructure.
\begin{df}
  A \textit{subring} of a ring \(R\) is a subset of \(R\) which inherits the ring structure and \(S \hookrightarrow R\) is a homomorphism
\end{df}

\subsection{The endomorphism group}

Let \(G\) be an abelian group, and let \(\mathrm{End}_{Ab}(G)\) be the endomorphisms. Then this is a ring in the obvious way (addition is pointwse, multiplication is composition).

\begin{exa}
  Notice that \(\mathrm{End}_{\mathrm{AB}}(\Z) \cong \Z\) as rings, with the isomorphism given by evaluation at \(1\). 
\end{exa}

\subsection{Cayley's Theorem for Rings}

Cayley's theorem, which states that every group is a subgroup of a symmetric group, in particular states that a group acts on itself by left-multiplication. Now we define

\begin{df}
  Take \(r \in R\), then let \(\lambda_r:R \to R\) be given by \(\lambda_r(s) = rs\).
\end{df}

\begin{thm}
  Let \(R\) be a ring. Then \(r \xmapsto{\lambda} \lambda_r\) is an injective ring homomorphism.
\end{thm}

The proof of this statement is actually just boiling down to chcking that the map is indeed an injective homomorphism. Nothing new. We've admitted it.

\subsection{Ideals}
Ideals are the analogue of normal subgroups for rings.

\begin{df}
  An ideal \(I\) of a ring \(R\)  is a subgroup such that
  \begin{align*}
    lI \subseteq I \quad (\forall l \in R) \\
    Ir \subseteq I \quad (\forall r \in R)
  \end{align*}

  Taking only one of these two conditions will give the definition of a left or right ideal respectively.
\end{df}

So now we can study what's so nice about ideals:

\begin{thm}
  Let \(\phi: R \to S\) be a ring homomorphism. Then \(\mathrm{ker} \phi\) is an ideal.
\end{thm}

\subsection{Quotients}

We define a quotient ring in the same way that we define quotient groups on groups, simply replacing our language of normal subgroups with our language of ideals.

\begin{df}
  Let \(I \leq R\) be an ideal. Then \(R / I\) is an abelian group with
  \begin{align*}
    (r + I) + (s + I) = (r + s) + I = (s + r) + I
  \end{align*}

  and a ring with
  \begin{align*}
    (r + I) \cdot (s + I) = r \cdot s + I
  \end{align*}
\end{df}

Showing that ths is well defined is no trick.

Suppose that \(r_1 - r_2 \in I,\) and \(s_1 - s_2 \in I\), in other words, \(r_1\) and \(r_2\) are in the same coset of our ideal (and for \(s\)). Then consider \(r_1s_1 - r_2s_2 = r_1(s_1 - s_2) + (r_1 - r_2)s_2\), so we have well definedness.

\begin{exa}
  \(\Z / n\Z\) is a ring.

  Consider the cannonical map
  \begin{align*}
    \Z \to R \\
    n \mapsto \underbrace{1_R + \ldots + 1_R}_{n\ \mathrm{times}}
  \end{align*}

  the kernel of this map is an ideal, and is either \(n\Z\) or 0
\end{exa}
From this example, there is a natural definition
\begin{df}
  As in the example above, \(n\) is the \textit{characteristic} of \(R\). If the ideal generated is the 0 ideal, then the ring is characteristic 0.
\end{df}

\begin{thm}
  Let \(I\) be an ideal of \(R\). Then for every ring homomorphism \(\phi: R \to S\) such that \(I \subseteq \mathrm{ker}\phi\), there is a unique ring homomorphism \(\overline{\phi}\) such that the diagram

  \begin{center}
    \begin{tikzcd}
      R \rar{\phi} & S \dar[leftarrow]{\overline{\phi}}\\
      & R/I  \ular[leftarrow]{\pi}
    \end{tikzcd}
  \end{center}
  commutes.
\end{thm}

Then just like for groups, we get that if \(\phi: R \to S\) is a ring homomorphism, then

\begin{center}
  \begin{tikzcd}
    R \arrow[twoheadrightarrow, r] \arrow[rrr, bend right] & R/\textrm{ker}(\phi) \arrow[r, "\cong"] & \lar[leftarrow, "\phi"] \textrm{im} \phi \arrow[r, hookrightarrow] & S
  \end{tikzcd}
\end{center}
and hence the first isomorphism theorem for rings:

\begin{thm}
  Let \(\phi: R \to S\) be a surjective map, then
  \begin{align*}
    R / \mathrm{ker} \phi \cong S
  \end{align*}
\end{thm}
\end{document}
%%% Local Variables:
%%% mode: latex
%%% TeX-master: t
%%% End:

%%% Local Variables:
%%% mode: latex
%%% TeX-master: t
%%% End:



