\documentclass[12pt, twosided]{article}
\usepackage[letterpaper,bindingoffset=0in,%
            left=1in,right=1in,top=1in,bottom=1in,%
            footskip=.25in]{geometry}

\usepackage{mathtools}
\usepackage{graphicx}

%\usepackage{setspace}
%\setstretch{1.1}

\usepackage{amsmath}
\usepackage{amsfonts}
\usepackage{amsthm}
\usepackage{amssymb}
\usepackage{csquotes}
\usepackage{relsize}

\usepackage{tikz}
\usetikzlibrary{cd}
\usetikzlibrary{fit,shapes.geometric}
\tikzset{%  
    mdot/.style={draw, circle, fill=black},
    mset/.style={draw, ellipse, very thick},
}

\usepackage{hhline}
\usepackage{systeme}
\usepackage{mathrsfs}
\usepackage{hyperref}
\usepackage{mathtools}  
\usepackage{silence}
\usepackage{blkarray}
\usepackage{float}
% \usepackage{framed}
\usepackage{mdframed}
\usepackage{array}
\usepackage{stmaryrd}
\usepackage{extarrows}
\usepackage{caption}
\captionsetup[figure]{labelfont={bf},name={Fig.},labelsep=period}

\makeatletter
\newtheoremstyle{indentedDf}
  {8pt}% space before
  {8pt}% space after
  {\addtolength{\@totalleftmargin}{3.5em}
   \addtolength{\linewidth}{-7em}
   \parshape 1 3.5em \linewidth}% body font
  {}% indent
  {\bfseries}% header font
  {.}% punctuation
  {.5em}% after theorem header
  {}% header specification (empty for default)
\makeatother

\makeatletter
\newtheoremstyle{indentedPlain}
  {8pt}% space before
  {8pt}% space after
  {\itshape
   \addtolength{\@totalleftmargin}{3.5em}
   \addtolength{\linewidth}{-7em}
   \parshape 1 3.5em \linewidth}% body font
  {}% indent
  {\bfseries}% header font
  {.}% punctuation
  {.5em}% after theorem header
  {}% header specification (empty for default)
\makeatother

\theoremstyle{indentedDf}
\newtheorem{df}{Definition}[section]
\newtheorem{exa}[df]{Example}
\newtheorem{ques}[df]{Question}
\newtheorem{exr}[df]{Exercise}
\newtheorem{prb}[df]{Problem}
\newtheorem*{notn}{Notation}
\newtheorem*{note}{Note}

\theoremstyle{indentedPlain}
\newtheorem{thm}[df]{Theorem}
\newtheorem{prop}[df]{Proposition}
\newtheorem{conj}[df]{Conjecture}
\newtheorem{cor}[df]{Corollary}
\newtheorem{lm}[df]{Lemma}
\newtheorem*{fact}{Fact}
\newtheorem*{idea}{Idea}
\newtheorem*{clm}{Claimn}
\newtheorem*{rmk}{Remark}
\usepackage[ruled]{algorithm2e}

\usepackage{ulem}
\makeatletter

\def\lf{\left\lfloor}   
\def\rf{\right\rfloor}
\def\lc{\left\lceil}   
\def\rc{\right\rceil}
\def\st{\text{ s.t. }}
\def\ker{\mathrm{ker}\ }
\def\im{{\mathrm{im}}\ }
\def\1{^{-1}}
\def\2{^2}
\def\3{^3}
\def\tn{^n}
\def\ind{\mathbf{1}}
\def\R{\mathbb{R}}
\def\Q{\mathbb{Q}}
\def\Z{\mathbb{Z}}
\def\C{\mathbb{C}}
\def\I{\mathbb{I}}
\def\N{\mathbb{N}}
\def\F{\mathbb{F}}
\def\A{\mathbb{A}}
\def\Li{\text{Li}}
\def\th{^\text{th}}
\def\sp{\text{Sp}}
\def\opn{\left\{}
\def\cls{\right\}}
\def\Aut{\text{Aut}}
\def\PG{\text{PG}}
\def\GL{\text{GL}}
\def\PGL{\text{PGL}}
\def\Cov{\text{Cov}}
\def\Pack{\text{Pack}}
\def\PgamL{\text{P}\Gamma\text{L}}
\def\gamL{\Gamma\text{L}}
\def\cl{\text{cl}}
\def\stbar{\ \middle\vert\ }
\def\partdone{\hphantom{1} \hfill \(\triangle\)}
\def\s0{_0}
\def\s1{_1}
\def\s2{_2}
\def\id{\mathrm{id}}
\def\topn{\text{ open}}
\def\Bd{\text{Bd }}
\def\nope{\(\longrightarrow\!\!\longleftarrow\)}
\def\stt{\(^{\text{st}}\)}
\def\tht{\(^{\text{th}}\)}
\def\ndt{\(^{\text{nd}}\)}
\def\t{^{T}}
\def\c{^c}
\newcommand\norm[1]{\left\lVert#1\right\rVert}
\renewcommand{\P}{\mathbb{P}}
\newcommand{\leg}[2]{\left( \frac{#1}{#2} \right)}

\renewcommand*\env@matrix[1][*\c@MaxMatrixCols c]{%
   \hskip -\arraycolsep
   \let\@ifnextchar\new@ifnextchar
   \array{#1}}
\makeatother

% These two lines suppress the warning generated 
% by amsmath for overwriting the choose command  
% because it's annoying. This probably has unint-
% ended ramifications somewhere else, but I'm too
% lazy to actually figure that out, so we'll cro-
% ss that bridge when we come to it lol.
\renewcommand{\choose}[2]{\left( {#1 \atop #2} \right)}
\WarningFilter{amsmath}{Foreign command} 

\renewcommand{\mod}[1]{\ (\mathrm{mod}\ #1)}
\renewcommand{\vec}[1]{\mathbf{#1}}

\let\oldprime\prime
\def\prime{^\oldprime}

\usepackage{float}
\restylefloat{figure}

\usepackage{cleveref}
\Crefname{thm}{Theorem}{Theorems}

% Comment commands for co-authors
\newcommand{\kmd}[1]{{\color{purple} #1}}

\newcolumntype{L}{>{$}l<{$}}
% Bib matter
\let\oldepsilon\epsilon
\def\epsilon{\varepsilon}

\let\oldphi\phi
\def\phi{\varphi}

% New definition of square root:
% it renames \sqrt as \oldsqrt
\let\oldsqrt\sqrt
% it defines the new \sqrt in terms of the old one
\def\sqrt{\mathpalette\DHLhksqrt}
\def\DHLhksqrt#1#2{%
\setbox0=\hbox{$#1\oldsqrt{#2\,}$}\dimen0=\ht0
\advance\dimen0-0.2\ht0
\setbox2=\hbox{\vrule height\ht0 depth -\dimen0}%
{\box0\lower0.4pt\box2}}
%%% Local Variables:
%%% mode: plain-tex
%%% TeX-master: t
%%% End:

\graphicspath{{./img/}}

\begin{document}
\noindent \textbf{Math 245} \hfill \textbf{Genevieve Walsh} \\
\textbf{Kyle Dituro} \hfill \textbf{January 22\ndt, 2023}\hrule
\vspace{.2in}

We discuss a little bit more before moving on to ``abstract nonsense''.

Recall that we have the universal property of the product, where something is a product if maps to projections factor through the product. Diagrammatically, create a category whose morphisms are the \(\sigma\) such that the below diagram commutes,

\begin{center}
  \begin{tikzcd}
    & Z_1 \arrow[dl, "g_1"] \arrow[d, "\sigma"] \arrow[dr, "f_1"] & \\
    B & Z_2 \arrow[l, "g_2"] \arrow[r, "f_2"] & A
  \end{tikzcd}
\end{center}

and the objects are things of the type
\begin{center}
  \begin{tikzcd}
    & Z \arrow[dl, "g"] \arrow[dr, "f"] & \\
    B & & A
  \end{tikzcd}
\end{center}

Then notice that, in this category, the product

\begin{center}
  \begin{tikzcd}
    & A \times B \arrow[dl, "\pi_B"] \arrow[dr, "\pi_A"] & \\
    B & & A
  \end{tikzcd}
\end{center}

Is final, with morphisms

\begin{center}
  \begin{tikzcd}
    & Z_1 \arrow[dl, "g"] \arrow[d, "\sigma"] \arrow[dr, "f"] & \\
    B & A \times B \arrow[l, "\pi_B"] \arrow[r, "\pi_A"] & A
  \end{tikzcd}
\end{center}

Where \(\sigma(z) = (f(z), g(z))\). Here it is clear that the product map is unique, so this truly is final.

\begin{df}
  We say that a category \(\mathcal{C}\) \textit{has products} if \(\forall A, B \in \mathrm{Obj}(\mathcal{C})\), then \(\mathcal{C}_{A, B}\) has a final object. in SET, final objects are disjoint unions.
\end{df}

\section{Groups}

\begin{df}[Group]
  \(G\) is a set with a closed binary operation
  \begin{align*}
    (\cdot): G \times G \to G \quad \cdot (g, h) \mapsto g \times h
  \end{align*}
  such that
  \begin{enumerate}
  \item \(\cdot\) is associative
  \item \(\exists e_G \in G\) such that \(\forall g \in G\) \(g \cdot e = e \cdot g = g\)
  \item We get inverses: \(\forall g \in G\), there exists \(g\1 \in G\) such that \(g \cdot g\1 = g\1 \cdot g = e_G\).
  \end{enumerate}
\end{df}

\begin{exa}
  The trivial group. Yep.
\end{exa}
\begin{exa}
  \((\Z, +)\), \((\C, +)\), \((\pm 1, \cdot)\). Notice that all of these groups are abelian.
\end{exa}

\begin{exa}
  Matrix groups: \((\mathrm{GL}_n(F),\ \cdot)\), the multiplicative group of invertible \(n \times n\) matrices over a field. Notice that matrix multiplication doesn't commute: a trivial fact.
\end{exa}

\begin{prop}
  Both the identity and \(g\1\) are unique.
\end{prop}
\begin{proof}
  Trivial. Suppose \(\exists e_1, e_2\) both identities for the sake of contradiction. But then their product blah blah blah. And likewise for inverses.
\end{proof}
  \begin{notn}
    By associativity, we can justify that
    \begin{align*}
      g^n = \underbrace{g \cdot g \cdot \ldots \cdot g}_{n \mathrm{times}}
    \end{align*}
  \end{notn}

  As always, a group is called \textit{Abelian} if the binary operation is commutative.

  \begin{exa}
    The Commutator group for a non-abelian group is such that the binary op. is
    \begin{align*}
      [a, b] = aba\1b\1
    \end{align*}

    Likewise the commutator subgroup is a subgroup generated by all commutators of elements from \(G\).
  \end{exa}

  \begin{df}
    The order of an element \(g \in G\) is finite if \(\exists n \in \N\) such that \(g^n = \mathrm{Id}\). The order of the element is the least such \(n\). An element has infinite order otherwise.
  \end{df}

  \begin{lm}
    If \(g^n = e\) for some \(n > 0\), then \(|g|\) divides \(n\).
  \end{lm}
  \begin{proof}
    Trivial exercise in number theory.
  \end{proof}

  \begin{cor}
    \begin{align*}
      g^N \Leftrightarrow N \text{ is a multiple of } |g|.
    \end{align*}
  \end{cor}

  \begin{df}
    \(|G|\) is the number of elements in \(G\), potentially \(\infty\).
  \end{df}

  \begin{prop}
    If \(gh = hg\) then \(|gh|\) divides \(\mathrm{lcm}(|g|, |h|)\)
  \end{prop}
  \begin{proof}
    Evaluation.
  \end{proof}

  \begin{exa}
    The Symmetric group \(S_n\) with order \(n!\). Notice that permutational composition goes in lexicographic order, which stinks. Also obviously \(S_n\) is not generally (in fact hardly ever) abelian. If so prompted, we can investigate the structure of \(S_3\), and write down a presentation of 6 generators.
  \end{exa}

  \begin{exa}
    The Dihedral groups \(D_n\) are the groups that are isometries of a regular \(n\)-sided polygon\footnote{\url{https://www.youtube.com/watch?v=fV7zFzhqYps}} in \(\R^2\).
  \end{exa}

  \begin{exa}
    Also of course we have the cyclic groups \(\Z / n\Z\) which we will write as cosets under addition of cosets.

    Likewise we can define
    \(\left( \Z / p\Z \right)^*\), the multiplicative group mod \(p\). This is a group because trust me. Notice that it's kinda hard to find generators for this in general. It is sometimes cyclic.
  \end{exa}

  Notice that groups together with group homomorphisms form a category called GRP.

  \begin{center}
    \begin{tikzcd}
      G \times G \arrow[d, "\cdot_G"] \arrow[r, "\phi \times \phi"] & H \times H \arrow[d, "\cdot_H"] \\
      G \arrow[r, "\phi"] & H
    \end{tikzcd}
  \end{center}
\end{document}
%%% Local Variables:
%%% mode: latex
%%% TeX-master: t
%%% End:

%%% Local Variables:
%%% mode: latex
%%% TeX-master: t
%%% End:

