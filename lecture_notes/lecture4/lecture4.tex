\documentclass[12pt, twosided]{article}
\usepackage[letterpaper,bindingoffset=0in,%
            left=1in,right=1in,top=1in,bottom=1in,%
            footskip=.25in]{geometry}

\usepackage{mathtools}
\usepackage{graphicx}

%\usepackage{setspace}
%\setstretch{1.1}

\usepackage{amsmath}
\usepackage{amsfonts}
\usepackage{amsthm}
\usepackage{amssymb}
\usepackage{csquotes}
\usepackage{relsize}

\usepackage{tikz}
\usetikzlibrary{cd}
\usetikzlibrary{fit,shapes.geometric}
\tikzset{%  
    mdot/.style={draw, circle, fill=black},
    mset/.style={draw, ellipse, very thick},
}

\usepackage{hhline}
\usepackage{systeme}
\usepackage{mathrsfs}
\usepackage{hyperref}
\usepackage{mathtools}  
\usepackage{silence}
\usepackage{blkarray}
\usepackage{float}
% \usepackage{framed}
\usepackage{mdframed}
\usepackage{array}
\usepackage{stmaryrd}
\usepackage{extarrows}
\usepackage{caption}
\captionsetup[figure]{labelfont={bf},name={Fig.},labelsep=period}

\makeatletter
\newtheoremstyle{indentedDf}
  {8pt}% space before
  {8pt}% space after
  {\addtolength{\@totalleftmargin}{3.5em}
   \addtolength{\linewidth}{-7em}
   \parshape 1 3.5em \linewidth}% body font
  {}% indent
  {\bfseries}% header font
  {.}% punctuation
  {.5em}% after theorem header
  {}% header specification (empty for default)
\makeatother

\makeatletter
\newtheoremstyle{indentedPlain}
  {8pt}% space before
  {8pt}% space after
  {\itshape
   \addtolength{\@totalleftmargin}{3.5em}
   \addtolength{\linewidth}{-7em}
   \parshape 1 3.5em \linewidth}% body font
  {}% indent
  {\bfseries}% header font
  {.}% punctuation
  {.5em}% after theorem header
  {}% header specification (empty for default)
\makeatother

\theoremstyle{indentedDf}
\newtheorem{df}{Definition}[section]
\newtheorem{exa}[df]{Example}
\newtheorem{ques}[df]{Question}
\newtheorem{exr}[df]{Exercise}
\newtheorem{prb}[df]{Problem}
\newtheorem*{notn}{Notation}
\newtheorem*{note}{Note}

\theoremstyle{indentedPlain}
\newtheorem{thm}[df]{Theorem}
\newtheorem{prop}[df]{Proposition}
\newtheorem{conj}[df]{Conjecture}
\newtheorem{cor}[df]{Corollary}
\newtheorem{lm}[df]{Lemma}
\newtheorem*{fact}{Fact}
\newtheorem*{idea}{Idea}
\newtheorem*{clm}{Claimn}
\newtheorem*{rmk}{Remark}
\usepackage[ruled]{algorithm2e}

\usepackage{ulem}
\makeatletter

\def\lf{\left\lfloor}   
\def\rf{\right\rfloor}
\def\lc{\left\lceil}   
\def\rc{\right\rceil}
\def\st{\text{ s.t. }}
\def\ker{\mathrm{ker}\ }
\def\im{{\mathrm{im}}\ }
\def\1{^{-1}}
\def\2{^2}
\def\3{^3}
\def\tn{^n}
\def\ind{\mathbf{1}}
\def\R{\mathbb{R}}
\def\Q{\mathbb{Q}}
\def\Z{\mathbb{Z}}
\def\C{\mathbb{C}}
\def\I{\mathbb{I}}
\def\N{\mathbb{N}}
\def\F{\mathbb{F}}
\def\A{\mathbb{A}}
\def\Li{\text{Li}}
\def\th{^\text{th}}
\def\sp{\text{Sp}}
\def\opn{\left\{}
\def\cls{\right\}}
\def\Aut{\text{Aut}}
\def\PG{\text{PG}}
\def\GL{\text{GL}}
\def\PGL{\text{PGL}}
\def\Cov{\text{Cov}}
\def\Pack{\text{Pack}}
\def\PgamL{\text{P}\Gamma\text{L}}
\def\gamL{\Gamma\text{L}}
\def\cl{\text{cl}}
\def\stbar{\ \middle\vert\ }
\def\partdone{\hphantom{1} \hfill \(\triangle\)}
\def\s0{_0}
\def\s1{_1}
\def\s2{_2}
\def\id{\mathrm{id}}
\def\topn{\text{ open}}
\def\Bd{\text{Bd }}
\def\nope{\(\longrightarrow\!\!\longleftarrow\)}
\def\stt{\(^{\text{st}}\)}
\def\tht{\(^{\text{th}}\)}
\def\ndt{\(^{\text{nd}}\)}
\def\t{^{T}}
\def\c{^c}
\newcommand\norm[1]{\left\lVert#1\right\rVert}
\renewcommand{\P}{\mathbb{P}}
\newcommand{\leg}[2]{\left( \frac{#1}{#2} \right)}

\renewcommand*\env@matrix[1][*\c@MaxMatrixCols c]{%
   \hskip -\arraycolsep
   \let\@ifnextchar\new@ifnextchar
   \array{#1}}
\makeatother

% These two lines suppress the warning generated 
% by amsmath for overwriting the choose command  
% because it's annoying. This probably has unint-
% ended ramifications somewhere else, but I'm too
% lazy to actually figure that out, so we'll cro-
% ss that bridge when we come to it lol.
\renewcommand{\choose}[2]{\left( {#1 \atop #2} \right)}
\WarningFilter{amsmath}{Foreign command} 

\renewcommand{\mod}[1]{\ (\mathrm{mod}\ #1)}
\renewcommand{\vec}[1]{\mathbf{#1}}

\let\oldprime\prime
\def\prime{^\oldprime}

\usepackage{float}
\restylefloat{figure}

\usepackage{cleveref}
\Crefname{thm}{Theorem}{Theorems}

% Comment commands for co-authors
\newcommand{\kmd}[1]{{\color{purple} #1}}

\newcolumntype{L}{>{$}l<{$}}
% Bib matter
\let\oldepsilon\epsilon
\def\epsilon{\varepsilon}

\let\oldphi\phi
\def\phi{\varphi}

% New definition of square root:
% it renames \sqrt as \oldsqrt
\let\oldsqrt\sqrt
% it defines the new \sqrt in terms of the old one
\def\sqrt{\mathpalette\DHLhksqrt}
\def\DHLhksqrt#1#2{%
\setbox0=\hbox{$#1\oldsqrt{#2\,}$}\dimen0=\ht0
\advance\dimen0-0.2\ht0
\setbox2=\hbox{\vrule height\ht0 depth -\dimen0}%
{\box0\lower0.4pt\box2}}
%%% Local Variables:
%%% mode: plain-tex
%%% TeX-master: t
%%% End:

\graphicspath{{./img/}}

\begin{document}
\noindent \textbf{Math 245} \hfill \textbf{Genevieve Walsh} \\
\textbf{Kyle Dituro} \hfill \textbf{January 29\tht, 2024}\hrule
\vspace{.2in}

\section{Group Presentations}
We start out with an important (somewhat trivial) fact:

\begin{fact}
  Every group is the quotient of a free group.
\end{fact}

If \(G\) is a group, then we can surject
\begin{align*}
  F(G) \twoheadrightarrow G \\
  g_1g_2g_3 \longmapsto g_1 \times g_2 \times g_3.
\end{align*}
This sucks. We can do better. Often times, there exists a smaller set \(A \subseteq G\) such that

\begin{align*}
  F(A) \twoheadrightarrow G \\
  a \longmapsto a \\
  a_1a_2a_3 \longmapsto a_1 \times a_2 \times a_3.
\end{align*}

If this is a surjection, \(A\) is a set of generators, and if \(A\) is finite, then \(G\) is \textit{finitely generated.}

Now let 
\begin{center}
  \begin{tikzcd}
    R \rar & F(A) \rar[twoheadrightarrow] & G
  \end{tikzcd}
\end{center}
Be exact at \(F(A)\), then \(R\) is a \textit{normal} subgroup.

So let \(\mathcal{R}\) be the set of words such that \(\langle\langle \mathcal{R}\rangle\rangle\)

So we define \(B \subseteq G\) to be
\begin{align*}
  \langle\langle B \rangle\rangle_G &:= \bigcap N : N \trianglelefteq G \mathrm{\ and\ } B \subseteq N \\
  \langle\langle \mathcal{R} \rangle\rangle &= \text{The smallest normal subgroup that contains } \mathcal{R}.
\end{align*}

\begin{exa}
  Okay, so, like, don't worry about the fact that we haven't defined presentations yet.

  \begin{align*}
    D_{10} = \left\langle r,s \stbar r^5 = 1, s^2 = 1, sr = r^4s \right\rangle
  \end{align*}

  Being the dihedral group on 10 elements, and

  \begin{center}
    \begin{tikzcd}
      \langle\langle r^5, s^2, srsr \rangle\rangle \rar & F(r,s) \rar[twoheadrightarrow] & D_{10}
    \end{tikzcd}
  \end{center}
\end{exa}

\begin{df}
  \(G\) has a presentation
  \begin{align*}
    \left\langle A \stbar \mathcal{R} \right\rangle
  \end{align*}
  if \(F(A) \twoheadrightarrow G\) is a surjection and \(\langle\langle \mathcal{R} \rangle\rangle_{F(A)}\) is the kernel.

  If the kernel \(\langle\langle \mathcal{R} \rangle\rangle\) is written with \(\mathcal{R}\) a finite list of words and \(A\) is finite, then \(G = \left\langle A \stbar \mathcal{R}\right\rangle\)
\end{df}

If \(p\) is prime and \(|G| = p\), what is \(G\)? \(\exists x: x^p = 1\), and \(G = \langle x \vert x^p = 1 \rangle = \Z / p\Z\)

\begin{thm}[Cauchy's Theorem.]
  If \(p \mid |G|\) and \(p\) is prime, then there exists an element of order \(p\) in \(G\).
\end{thm}
\section{Group Actions}
Let \(A\) be a set, and \(G\) be a group. Then take a homomorphism \(G \to \mathrm{Aut}(A)\). Every group is a subgroup of a permutaton gp. \(G\) acts on \(G\) by left multiplication.

\begin{df}
  \(G \curvearrowright A\) is \textit{transitive} if \(\forall a, b \in A\), \(\exists g \in G\) such that \(g(a) = b\).
\end{df}

\begin{exa}
  \(G \curvearrowright G\) by conjugation is \textbf{not} a transitive action... especially clear in abelian groups. Notice that \(gag\1 = a\) in any abelian group, so the action is not transitive.
\end{exa}

\begin{df}
  The \textit{orbit} of \(a \in A\) under \(G\) is the set 
  \begin{align*}
    O_G(a) = \opn ga \stbar g\in G \cls \subseteq A.
  \end{align*}
\end{df}

Orbits partition the set \(A\).

\begin{df}
  The \textit{stabilizer subgroup} of \(a \in A\) is

  \begin{align*}
    \mathrm{Stab}_G(a) = \opn g \in G \stbar g(a) = a \cls \leq G.
  \end{align*}
\end{df}

We make a brief return to categories to make a few statements.

Let \(G\) be a group, call sets with a group action \(G\)-sets. Then consider the category whose objects are \((\rho, A)\), \(\rho: G \times A \to A\) such that


\begin{center}
  \begin{tikzcd}
    G \times A_1 \dar{\rho_1} \rar{(\text{Id}, \phi)} & G \times A_2 \dar{\rho_2} \\
    A_1 \rar{\phi} & A_2
  \end{tikzcd}
\end{center}
commutes. Such a \(\phi\) is called a \(G\)-equivariant function  if \(\forall g \in G\), \(g\phi(a) = \phi(ga)\).

Two \(G\) sets are called isomorphic if there is an equivariant bijection.

\begin{prop}
  Every transitive left action of \(G\) on a set is isomorphic to

  \begin{align*}
    \rho G \times G/H \to G/H \\
    \rho(g_1, g_2H) = g_1g_2H,
  \end{align*}
  Where \(H\) is the stabilizer of any \(a \in A\).
\end{prop}
\begin{proof}
  \(G \curvearrowright A\) transitively. \(H = \mathrm{Stab}_G(a)\).

  Then let
  \begin{align*}
    \phi: G/\mathrm{Stab}(a) \to A \\
    \phi: gH \to ga.
  \end{align*}

  We claim that \(\phi\) is a \(G\)-equivariant bijection, but first that it is well defined.

  Suppose that \(g_1H = g_2H\). then \(g_2\1g_1H = H\), so \(g_2\1g_1 \in H = \mathrm{Stab}(a)\).

  Then \(g_2(a) = g_2(g_2\1g_1a) = g_1a\). \partdone

  Then we aught show it's a bijection.

  Well \(a\prime \in A \Rightarrow a\prime = ga\) for some \(g \in G\) by the action of \(G\) being transitive. so take

  \begin{align*}
    a\prime \longrightarrow gH.
  \end{align*}

  if \(a\prime = g_1a\) and \(a\prime = g_2a\), then

  \begin{align*}
    g_1a = g_2a &\Rightarrow g_2\1g_1(a) = a \\
                & g_2\1g_1 \in H = \mathrm{Stab}(a) \\
                &g_1H = g_2H.
  \end{align*} \partdone

  Then also \(\phi\) is equivariant by definition. Namely

  \begin{align*}
    \phi(g\prime g H) = g\prime g(a) \\
    g\prime \phi(gH) = g\prime(ga) = g\prime g a.
  \end{align*}
\end{proof}

Now we introduce a thoerem that is incredibly useful for counting things in groups.
\begin{cor}[The Orbit-Stabilizer Theorem]
  Let \(G \curvearrowright A\).

  If \(O_a\) is the orbit of \(a \in A\), then

  \begin{align*}
    |O_a|\cdot |\mathrm{Stab}_G(a)| = |G|
  \end{align*}
\end{cor}

\begin{cor}
  \(|O_a|\) divides \(|G|\)
\end{cor}

\begin{proof}
  \(G\) acts transitibely on \(O_a\) for any \(a\) by definition. So

  \begin{align*}
    \left\vert G/\mathrm{Stab}_G(a)\right\vert = |O|
  \end{align*}

  Notice that the left hand side is the number of left cosets. In other words,
  \begin{align*}
    |O_a| = [G:\mathrm{Stab}_G(a)]
  \end{align*}
  so
  \begin{align*}
    |O_a| \cdot |\mathrm{Stab}_G(a)| = |G|
  \end{align*}
\end{proof}

Now we introduce a small theorem that is not so difficult to prove.

\begin{thm}
  If \(G \curvearrowright A\), \(g(a) = b\), then
  \begin{align*}
    \mathrm{Stab}_G(b) = g\mathrm{Stab}_G(a)g\1.
  \end{align*}
  In other words 
\end{thm}

\begin{proof} \textbf{Admitted.} I dunno, just work this diagram and work by conjugation. You should be able to track elements via conjugatoin to get both containments.
  \begin{center}
    \begin{tikzcd}
      a \rar[bend left]{g} \arrow[loop left] & b \arrow[loop right]
    \end{tikzcd}
  \end{center}
\end{proof}

Therefore, these stabilizers are the same size.

\begin{prop}
  Let \(S\) be a finite set, \(G \curvearrowright S\). Then
  \begin{align*}
    |S| = \sum_{a \in A} [G: G_a], \quad G_a = \mathrm{Stab}(a)
  \end{align*}
  Where \(A\) contains exactly one element from from each orbit.
\end{prop}
\begin{proof}
  The orbits partition \(S\).

  \begin{align*}
    |S| = \sum_{a \in A}|O|, \quad |G| = |O| \cdot |\mathrm{Stab}(a)|
  \end{align*}

  Now let's pull out the things with one orbit.

  \begin{align*}
    Z = \text{Number of elements with one on element in it's orbit} \\
    Z = \opn a \stbar [G:G_a] = 1 \cls. \\
    |S| = |Z| + \sum_{a \in A}[G: G_a]
  \end{align*}
  Where \(A\) has one element from each non-trivial orbit. When the action is conjugation...

  Let \(Z(G)\) denote the center of \(G\), namely

  \begin{align*}
    Z(G) = \opn g \in G: ga = ag\ \forall a \in G \cls \\
    Z(a) = \opn g \in G \stbar ga = ag \cls
  \end{align*}

  Then the center is a subgroup.

  \begin{align*}
    |G| = |Z(G)| + \sum_{a \in A-1}[G:Z(a)]
  \end{align*}
  Where \(A\) contains exactly one element from each nontrivial orbit.
\end{proof}
In particular,

\begin{align*}
  |G| = \sum a_i \quad \text{where each } a_i \mid |G|
\end{align*}
\begin{exa}
  When \(|G| = 6\), we only have the options \(6 = 6\), and \(6 = 1 + 2 + 3\), and in particular,
  \(p = p\) where \(p\) is prime.
\end{exa}
\end{document}
%%% Local Variables:
%%% mode: latex
%%% TeX-master: t
%%% End:

%%% Local Variables:
%%% mode: latex
%%% TeX-master: t
%%% End:
