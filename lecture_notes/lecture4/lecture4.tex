\documentclass[12pt, twosided]{article}
\input{sdpreamble}
\graphicspath{{./img/}}

\begin{document}
\noindent \textbf{Math 245} \hfill \textbf{Genevieve Walsh} \\
\textbf{Kyle Dituro} \hfill \textbf{January 29\tht, 2024}\hrule
\vspace{.2in}

\section{Group Presentations}
We start out with an important (somewhat trivial) fact:

\begin{fact}
  Every group is the quotient of a free group.
\end{fact}

If \(G\) is a group, then we can surject
\begin{align*}
  F(G) \twoheadrightarrow G \\
  g_1g_2g_3 \longmapsto g_1 \times g_2 \times g_3.
\end{align*}
This sucks. We can do better. Often times, there exists a smaller set \(A \subseteq G\) such that

\begin{align*}
  F(A) \twoheadrightarrow G \\
  a \longmapsto a \\
  a_1a_2a_3 \longmapsto a_1 \times a_2 \times a_3.
\end{align*}

If this is a surjection, \(A\) is a set of generators, and if \(A\) is finite, then \(G\) is \textit{finitely generated.}

Now let 
\begin{center}
  \begin{tikzcd}
    R \rar & F(A) \rar[twoheadrightarrow] & G
  \end{tikzcd}
\end{center}
Be exact at \(F(A)\), then \(R\) is a \textit{normal} subgroup.

So let \(\mathcal{R}\) be the set of words such that \(\langle\langle \mathcal{R}\rangle\rangle\)

So we define \(B \subseteq G\) to be
\begin{align*}
  \langle\langle B \rangle\rangle_G &:= \bigcap N : N \trianglelefteq G \mathrm{\ and\ } B \subseteq N \\
  \langle\langle \mathcal{R} \rangle\rangle &= \text{The smallest normal subgroup that contains } \mathcal{R}.
\end{align*}

\begin{exa}
  Okay, so, like, don't worry about the fact that we haven't defined presentations yet.

  \begin{align*}
    D_{10} = \left\langle r,s \stbar r^5 = 1, s^2 = 1, sr = r^4s \right\rangle
  \end{align*}

  Being the dihedral group on 10 elements, and

  \begin{center}
    \begin{tikzcd}
      \langle\langle r^5, s^2, srsr \rangle\rangle \rar & F(r,s) \rar[twoheadrightarrow] & D_{10}
    \end{tikzcd}
  \end{center}
\end{exa}

\begin{df}
  \(G\) has a presentation
  \begin{align*}
    \left\langle A \stbar \mathcal{R} \right\rangle
  \end{align*}
  if \(F(A) \twoheadrightarrow G\) is a surjection and \(\langle\langle \mathcal{R} \rangle\rangle_{F(A)}\) is the kernel.

  If the kernel \(\langle\langle \mathcal{R} \rangle\rangle\) is written with \(\mathcal{R}\) a finite list of words and \(A\) is finite, then \(G = \left\langle A \stbar \mathcal{R}\right\rangle\)
\end{df}

If \(p\) is prime and \(|G| = p\), what is \(G\)? \(\exists x: x^p = 1\), and \(G = \langle x \vert x^p = 1 \rangle = \Z / p\Z\)

\begin{thm}[Cauchy's Theorem.]
  If \(p \mid |G|\) and \(p\) is prime, then there exists an element of order \(p\) in \(G\).
\end{thm}
\section{Group Actions}
Let \(A\) be a set, and \(G\) be a group. Then take a homomorphism \(G \to \mathrm{Aut}(A)\). Every group is a subgroup of a permutaton gp. \(G\) acts on \(G\) by left multiplication.

\begin{df}
  \(G \curvearrowright A\) is \textit{transitive} if \(\forall a, b \in A\), \(\exists g \in G\) such that \(g(a) = b\).
\end{df}

\begin{exa}
  \(G \curvearrowright G\) by conjugation is \textbf{not} a transitive action... especially clear in abelian groups. Notice that \(gag\1 = a\) in any abelian group, so the action is not transitive.
\end{exa}

\begin{df}
  The \textit{orbit} of \(a \in A\) under \(G\) is the set 
  \begin{align*}
    O_G(a) = \opn ga \stbar g\in G \cls \subseteq A.
  \end{align*}
\end{df}

Orbits partition the set \(A\).

\begin{df}
  The \textit{stabilizer subgroup} of \(a \in A\) is

  \begin{align*}
    \mathrm{Stab}_G(a) = \opn g \in G \stbar g(a) = a \cls \leq G.
  \end{align*}
\end{df}

We make a brief return to categories to make a few statements.

Let \(G\) be a group, call sets with a group action \(G\)-sets. Then consider the category whose objects are \((\rho, A)\), \(\rho: G \times A \to A\) such that


\begin{center}
  \begin{tikzcd}
    G \times A_1 \dar{\rho_1} \rar{(\text{Id}, \phi)} & G \times A_2 \dar{\rho_2} \\
    A_1 \rar{\phi} & A_2
  \end{tikzcd}
\end{center}
commutes. Such a \(\phi\) is called a \(G\)-equivariant function  if \(\forall g \in G\), \(g\phi(a) = \phi(ga)\).

Two \(G\) sets are called isomorphic if there is an equivariant bijection.

\begin{prop}
  Every transitive left action of \(G\) on a set is isomorphic to

  \begin{align*}
    \rho G \times G/H \to G/H \\
    \rho(g_1, g_2H) = g_1g_2H,
  \end{align*}
  Where \(H\) is the stabilizer of any \(a \in A\).
\end{prop}
\begin{proof}
  \(G \curvearrowright A\) transitively. \(H = \mathrm{Stab}_G(a)\).

  Then let
  \begin{align*}
    \phi: G/\mathrm{Stab}(a) \to A \\
    \phi: gH \to ga.
  \end{align*}

  We claim that \(\phi\) is a \(G\)-equivariant bijection, but first that it is well defined.

  Suppose that \(g_1H = g_2H\). then \(g_2\1g_1H = H\), so \(g_2\1g_1 \in H = \mathrm{Stab}(a)\).

  Then \(g_2(a) = g_2(g_2\1g_1a) = g_1a\). \partdone

  Then we aught show it's a bijection.

  Well \(a\prime \in A \Rightarrow a\prime = ga\) for some \(g \in G\) by the action of \(G\) being transitive. so take

  \begin{align*}
    a\prime \longrightarrow gH.
  \end{align*}

  if \(a\prime = g_1a\) and \(a\prime = g_2a\), then

  \begin{align*}
    g_1a = g_2a &\Rightarrow g_2\1g_1(a) = a \\
                & g_2\1g_1 \in H = \mathrm{Stab}(a) \\
                &g_1H = g_2H.
  \end{align*} \partdone

  Then also \(\phi\) is equivariant by definition. Namely

  \begin{align*}
    \phi(g\prime g H) = g\prime g(a) \\
    g\prime \phi(gH) = g\prime(ga) = g\prime g a.
  \end{align*}
\end{proof}

Now we introduce a thoerem that is incredibly useful for counting things in groups.
\begin{cor}[The Orbit-Stabilizer Theorem]
  Let \(G \curvearrowright A\).

  If \(O_a\) is the orbit of \(a \in A\), then

  \begin{align*}
    |O_a|\cdot |\mathrm{Stab}_G(a)| = |G|
  \end{align*}
\end{cor}

\begin{cor}
  \(|O_a|\) divides \(|G|\)
\end{cor}

\begin{proof}
  \(G\) acts transitibely on \(O_a\) for any \(a\) by definition. So

  \begin{align*}
    \left\vert G/\mathrm{Stab}_G(a)\right\vert = |O|
  \end{align*}

  Notice that the left hand side is the number of left cosets. In other words,
  \begin{align*}
    |O_a| = [G:\mathrm{Stab}_G(a)]
  \end{align*}
  so
  \begin{align*}
    |O_a| \cdot |\mathrm{Stab}_G(a)| = |G|
  \end{align*}
\end{proof}

Now we introduce a small theorem that is not so difficult to prove.

\begin{thm}
  If \(G \curvearrowright A\), \(g(a) = b\), then
  \begin{align*}
    \mathrm{Stab}_G(b) = g\mathrm{Stab}_G(a)g\1.
  \end{align*}
  In other words 
\end{thm}

\begin{proof} \textbf{Admitted.} I dunno, just work this diagram and work by conjugation. You should be able to track elements via conjugatoin to get both containments.
  \begin{center}
    \begin{tikzcd}
      a \rar[bend left]{g} \arrow[loop left] & b \arrow[loop right]
    \end{tikzcd}
  \end{center}
\end{proof}

Therefore, these stabilizers are the same size.

\begin{prop}
  Let \(S\) be a finite set, \(G \curvearrowright S\). Then
  \begin{align*}
    |S| = \sum_{a \in A} [G: G_a], \quad G_a = \mathrm{Stab}(a)
  \end{align*}
  Where \(A\) contains exactly one element from from each orbit.
\end{prop}
\begin{proof}
  The orbits partition \(S\).

  \begin{align*}
    |S| = \sum_{a \in A}|O|, \quad |G| = |O| \cdot |\mathrm{Stab}(a)|
  \end{align*}

  Now let's pull out the things with one orbit.

  \begin{align*}
    Z = \text{Number of elements with one on element in it's orbit} \\
    Z = \opn a \stbar [G:G_a] = 1 \cls. \\
    |S| = |Z| + \sum_{a \in A}[G: G_a]
  \end{align*}
  Where \(A\) has one element from each non-trivial orbit. When the action is conjugation...

  Let \(Z(G)\) denote the center of \(G\), namely

  \begin{align*}
    Z(G) = \opn g \in G: ga = ag\ \forall a \in G \cls \\
    Z(a) = \opn g \in G \stbar ga = ag \cls
  \end{align*}

  Then the center is a subgroup.

  \begin{align*}
    |G| = |Z(G)| + \sum_{a \in A-1}[G:Z(a)]
  \end{align*}
  Where \(A\) contains exactly one element from each nontrivial orbit.
\end{proof}
In particular,

\begin{align*}
  |G| = \sum a_i \quad \text{where each } a_i \mid |G|
\end{align*}
\begin{exa}
  When \(|G| = 6\), we only have the options \(6 = 6\), and \(6 = 1 + 2 + 3\), and in particular,
  \(p = p\) where \(p\) is prime.
\end{exa}
\end{document}
%%% Local Variables:
%%% mode: latex
%%% TeX-master: t
%%% End:

%%% Local Variables:
%%% mode: latex
%%% TeX-master: t
%%% End:
