\documentclass[12pt, twosided]{article}
\input{sdpreamble}
\graphicspath{{./img/}}

\begin{document}
\noindent \textbf{CLASS} \hfill \textbf{PROF} \\
\textbf{Kyle Dituro} \hfill \textbf{Month Day\tht, Year}\hrule
\vspace{.2in}

\section{Ideals}

They have generating sets, which is nice.

Now let's just talk about commutative rings. Then we let \(\langle a \rangle\) denote the ideal generated by \(a\). If an ideal is generated by a single element, we say that it is a \textit{principal ideal}. Moreover, if, for some ring \(R\), every ideal is principal and \(R\) has no zero divisors, then we say that \(R\) is a \textit{principal ideal domain}.

\begin{exa}
  \(\Z\) is a principal ideal domain.
\end{exa}

let \(I_\alpha\) be a faimily ideals indexed by some set \(\Lambda\). Then
\begin{align*}
  \sum_{\alpha \in \Lambda} I_\alpha
\end{align*}
is also an ideal. Moreover, if the ideals are finitely generated, then the sum is as well.

\begin{thm}
  \((R/\langle a \rangle) / \langle \overline b \rangle \cong R / \langle a, b \rangle\). This is one of the isomorphism theorems. I think that it's the second one.
\end{thm}

\begin{df}
  A commutative ring is \textit{Noetherian} if every ideal is finitely generated.
\end{df}

\begin{prop}
  Assume \(R\) is a finite commutative ring. Then \(R\) is an integral domain iff it is a field. (Notice that in the infinite case, \(\Z\) disproves this)
\end{prop}

This is far, far easier to check then the field condition, since inverses are -- in general -- hard to find.

So how else can we build ideals? Well we can take products and intersections of ideals that we already have.

\kmd{Yeah, sorry, I'm tired. I can't typeset this today. If you were counting on these notes, sorry I let you down.}
\end{document}
%%% Local Variables:
%%% mode: latex
%%% TeX-master: t
%%% End:

%%% Local Variables:
%%% mode: latex
%%% TeX-master: t
%%% End:
