\documentclass[12pt, twosided]{article}
\input{sdpreamble}
\usepackage{marvosym}
\graphicspath{{./img/}}

\begin{document}
\noindent \textbf{Math 245} \hfill \textbf{} \\
\textbf{Kyle Dituro} \hfill \textbf{February 5\tht, 2024}\hrule
\vspace{.2in}

\section{What's a Ring}
I dunno.

\begin{df}
  A ring \((R, +, \cdot)\) is an abelian group \((R, +)\) along with a second binary operation \((\cdot)\) which is associative and distributive over addition. (and with \(1\)).
\end{df}

\begin{notn}
  Call the additive identity \(0\) and the multiplicative identity 1.
\end{notn}

\begin{prop}
  \(\forall r \in R\),
  \begin{align*}
    r \cdot 0 = 0 \cdot r = 0 \\
    r \cdot 0 = r\cdot (0 + 0) = r \cdot 0 + r \cdot 0 \\
    \Rightarrow r \cdot 0 = 0
  \end{align*}
\end{prop}

\begin{fact}
  The additive inverse of \(r\) is \((-1 \cdot r)\)
\end{fact}

We sure are doing rings here. Man I love rings. Also, I learned about a new \LaTeX package, check this out:
{\huge \fax\ \fax\ \fax} We've even got \Faxmachine\ and \FAX. lol.

So anyway:

\begin{df}
  We say a ring has \textit{multiplicative cancellation} if \(\forall b \neq 0\) and \(a \cdot b = c \cdot b\) then \(a = c\).
\end{df}
\begin{exa}
  This isn't always true. In fact, notice \(\Z/ 6\Z\) does not have multiplicative cancellation.
\end{exa}

\begin{df}
  \(a \in R\) is a \textit{zero divisor} if \(\exists b \in R, b\neq 0\) such that \(a\cdot b = 0\)
\end{df}

\begin{prop}[The worst proposition]
  \(a \in R\) is not a zero divisor iff \(a:R \to R\) by \(a(b) = a \cdot b\) s injective.
\end{prop}


\begin{df}
  An \textit{integral domain} is a non-zero commutative ring without zero divisors.
\end{df}

So which cyclic rings are integral domains? Well notice that when \(n\) is nonprime, \(\Z/n\Z\) is not an integral domain.

Now let's look at some really wacky rings.

\begin{exa}
  Take \(S\) a set, and let \(\mathcal{P}(S)\) be the power set of \(S\). We can make this a ring with \(A \cup B \setminus (A \cap B)\) as addition. Then the additive identity of this ring is the empty set, and a set's inverse is the set complement.

  Then we define multiplication (which will also turn out to be commutative) as \(A \cdot B = A \cap B\), whose identity is the whole set, and whose distributativity is a bit trick to prove.

  \textbf{Hint:} Try drawing a Venn diagram or two.
\end{exa}

Here's another nasty one:

\begin{exa}
  Let \(R\) be a ring, and \(S\) be a set.

  Then \(R^S\) is the function ring of functions \(S \to R\), with \((f + g)(s) = f(s) + g(s)\) and \((f \cdot g)s = f(s)g(s)\). Then the additive identity is : \(f(s) = 0_R\) and multiplicative is \(f(s) = 1_R\). Then this ring has zero divisors in the form of function pairs which take a nonzero value whenever the other takes zero.
\end{exa}

Then something is a (left or right) unit if \(\exists v\) such that \(u \cdot v = 1\) (or visa-versa). A division ring is is a ring where every non zero element is a two sided unit.

\begin{df}
  A \textit{Field} is a non zero commutative ring where every nonzero element is a unit.
\end{df}

\section{Polynomial Rings}

Let \(R\) be a ring. then \(R[x]\) is the set of polynomials in indeterminate \(x\) (consider only polynomials with finitely many terms). Then if \(f = \sum a_ix^i\) and \(g = \sum b_ix^i\), then \(f + g = \sum(a_i + b_i)x^i\), and \(fg = \sum_{k \in \N}\sum_{i + j}a_ib_jx^k\). This is a ring. Have fun checking this you fucks lmao. Now in the same vein, \(R[[x]]\) is the ring of formal power series in \(R\).

\section{Ring Homomorphisms}

\begin{df}
  A \textit{homomorphism of rings} is a map \(\phi\) which is homomorphic on both the addition and the multiplication, and takes the identity of one category to the other.
\end{df}

And naturally an isomorphism is a homomorphism with rings.

Now we can talk about the category or rings, RING. thank god. Then in this category, \(\Z\) is initial. namely, we can construct a homomorphism \(\Z \to R\) which maps \(1 \mapsto 1_R\), \(-1 \mapsto -1_R\), and \(m \mapsto \underbrace{1_R + \ldots + 1_R}_{m\ \mathrm{times}} = m1_R\). Notice that the additive homomorphism falls out trivially. Then in order to get multiplication in general, we need to get that
\begin{align*}
  \phi(m)\phi(n) &= m1_R \cdot n1_R \\
                 &= m \cdot n 1_R 1_R = mn 1_R \\
                 &=\phi(m \cdot n).
\end{align*}

\begin{cor}
  Ring homomorphisms take units to units.
\end{cor}

Now we give a particularly interesting example, despite the fact that rings on their own are boring as shit.

\begin{exa}
  \(\Z[X_1, \ldots, X_n]\), the set of polynomials in some number \(n\) of indeterminants, is the analogue of free groups. And in particular this can be verified using the category \(\mathcal{R}_A\).
\end{exa}
\end{document}
%%% Local Variables:
%%% mode: latex
%%% TeX-master: t
%%% End:

%%% Local Variables:
%%% mode: latex
%%% TeX-master: t
%%% End:
