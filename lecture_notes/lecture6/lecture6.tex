\documentclass[12pt, twosided]{article}
\usepackage[letterpaper,bindingoffset=0in,%
            left=1in,right=1in,top=1in,bottom=1in,%
            footskip=.25in]{geometry}

\usepackage{mathtools}
\usepackage{graphicx}

%\usepackage{setspace}
%\setstretch{1.1}

\usepackage{amsmath}
\usepackage{amsfonts}
\usepackage{amsthm}
\usepackage{amssymb}
\usepackage{csquotes}
\usepackage{relsize}

\usepackage{tikz}
\usetikzlibrary{cd}
\usetikzlibrary{fit,shapes.geometric}
\tikzset{%  
    mdot/.style={draw, circle, fill=black},
    mset/.style={draw, ellipse, very thick},
}

\usepackage{hhline}
\usepackage{systeme}
\usepackage{mathrsfs}
\usepackage{hyperref}
\usepackage{mathtools}  
\usepackage{silence}
\usepackage{blkarray}
\usepackage{float}
% \usepackage{framed}
\usepackage{mdframed}
\usepackage{array}
\usepackage{stmaryrd}
\usepackage{extarrows}
\usepackage{caption}
\captionsetup[figure]{labelfont={bf},name={Fig.},labelsep=period}

\makeatletter
\newtheoremstyle{indentedDf}
  {8pt}% space before
  {8pt}% space after
  {\addtolength{\@totalleftmargin}{3.5em}
   \addtolength{\linewidth}{-7em}
   \parshape 1 3.5em \linewidth}% body font
  {}% indent
  {\bfseries}% header font
  {.}% punctuation
  {.5em}% after theorem header
  {}% header specification (empty for default)
\makeatother

\makeatletter
\newtheoremstyle{indentedPlain}
  {8pt}% space before
  {8pt}% space after
  {\itshape
   \addtolength{\@totalleftmargin}{3.5em}
   \addtolength{\linewidth}{-7em}
   \parshape 1 3.5em \linewidth}% body font
  {}% indent
  {\bfseries}% header font
  {.}% punctuation
  {.5em}% after theorem header
  {}% header specification (empty for default)
\makeatother

\theoremstyle{indentedDf}
\newtheorem{df}{Definition}[section]
\newtheorem{exa}[df]{Example}
\newtheorem{ques}[df]{Question}
\newtheorem{exr}[df]{Exercise}
\newtheorem{prb}[df]{Problem}
\newtheorem*{notn}{Notation}
\newtheorem*{note}{Note}

\theoremstyle{indentedPlain}
\newtheorem{thm}[df]{Theorem}
\newtheorem{prop}[df]{Proposition}
\newtheorem{conj}[df]{Conjecture}
\newtheorem{cor}[df]{Corollary}
\newtheorem{lm}[df]{Lemma}
\newtheorem*{fact}{Fact}
\newtheorem*{idea}{Idea}
\newtheorem*{clm}{Claimn}
\newtheorem*{rmk}{Remark}
\usepackage[ruled]{algorithm2e}

\usepackage{ulem}
\makeatletter

\def\lf{\left\lfloor}   
\def\rf{\right\rfloor}
\def\lc{\left\lceil}   
\def\rc{\right\rceil}
\def\st{\text{ s.t. }}
\def\ker{\mathrm{ker}\ }
\def\im{{\mathrm{im}}\ }
\def\1{^{-1}}
\def\2{^2}
\def\3{^3}
\def\tn{^n}
\def\ind{\mathbf{1}}
\def\R{\mathbb{R}}
\def\Q{\mathbb{Q}}
\def\Z{\mathbb{Z}}
\def\C{\mathbb{C}}
\def\I{\mathbb{I}}
\def\N{\mathbb{N}}
\def\F{\mathbb{F}}
\def\A{\mathbb{A}}
\def\Li{\text{Li}}
\def\th{^\text{th}}
\def\sp{\text{Sp}}
\def\opn{\left\{}
\def\cls{\right\}}
\def\Aut{\text{Aut}}
\def\PG{\text{PG}}
\def\GL{\text{GL}}
\def\PGL{\text{PGL}}
\def\Cov{\text{Cov}}
\def\Pack{\text{Pack}}
\def\PgamL{\text{P}\Gamma\text{L}}
\def\gamL{\Gamma\text{L}}
\def\cl{\text{cl}}
\def\stbar{\ \middle\vert\ }
\def\partdone{\hphantom{1} \hfill \(\triangle\)}
\def\s0{_0}
\def\s1{_1}
\def\s2{_2}
\def\id{\mathrm{id}}
\def\topn{\text{ open}}
\def\Bd{\text{Bd }}
\def\nope{\(\longrightarrow\!\!\longleftarrow\)}
\def\stt{\(^{\text{st}}\)}
\def\tht{\(^{\text{th}}\)}
\def\ndt{\(^{\text{nd}}\)}
\def\t{^{T}}
\def\c{^c}
\newcommand\norm[1]{\left\lVert#1\right\rVert}
\renewcommand{\P}{\mathbb{P}}
\newcommand{\leg}[2]{\left( \frac{#1}{#2} \right)}

\renewcommand*\env@matrix[1][*\c@MaxMatrixCols c]{%
   \hskip -\arraycolsep
   \let\@ifnextchar\new@ifnextchar
   \array{#1}}
\makeatother

% These two lines suppress the warning generated 
% by amsmath for overwriting the choose command  
% because it's annoying. This probably has unint-
% ended ramifications somewhere else, but I'm too
% lazy to actually figure that out, so we'll cro-
% ss that bridge when we come to it lol.
\renewcommand{\choose}[2]{\left( {#1 \atop #2} \right)}
\WarningFilter{amsmath}{Foreign command} 

\renewcommand{\mod}[1]{\ (\mathrm{mod}\ #1)}
\renewcommand{\vec}[1]{\mathbf{#1}}

\let\oldprime\prime
\def\prime{^\oldprime}

\usepackage{float}
\restylefloat{figure}

\usepackage{cleveref}
\Crefname{thm}{Theorem}{Theorems}

% Comment commands for co-authors
\newcommand{\kmd}[1]{{\color{purple} #1}}

\newcolumntype{L}{>{$}l<{$}}
% Bib matter
\let\oldepsilon\epsilon
\def\epsilon{\varepsilon}

\let\oldphi\phi
\def\phi{\varphi}

% New definition of square root:
% it renames \sqrt as \oldsqrt
\let\oldsqrt\sqrt
% it defines the new \sqrt in terms of the old one
\def\sqrt{\mathpalette\DHLhksqrt}
\def\DHLhksqrt#1#2{%
\setbox0=\hbox{$#1\oldsqrt{#2\,}$}\dimen0=\ht0
\advance\dimen0-0.2\ht0
\setbox2=\hbox{\vrule height\ht0 depth -\dimen0}%
{\box0\lower0.4pt\box2}}
%%% Local Variables:
%%% mode: plain-tex
%%% TeX-master: t
%%% End:

\usepackage{marvosym}
\graphicspath{{./img/}}

\begin{document}
\noindent \textbf{Math 245} \hfill \textbf{} \\
\textbf{Kyle Dituro} \hfill \textbf{February 5\tht, 2024}\hrule
\vspace{.2in}

\section{What's a Ring}
I dunno.

\begin{df}
  A ring \((R, +, \cdot)\) is an abelian group \((R, +)\) along with a second binary operation \((\cdot)\) which is associative and distributive over addition. (and with \(1\)).
\end{df}

\begin{notn}
  Call the additive identity \(0\) and the multiplicative identity 1.
\end{notn}

\begin{prop}
  \(\forall r \in R\),
  \begin{align*}
    r \cdot 0 = 0 \cdot r = 0 \\
    r \cdot 0 = r\cdot (0 + 0) = r \cdot 0 + r \cdot 0 \\
    \Rightarrow r \cdot 0 = 0
  \end{align*}
\end{prop}

\begin{fact}
  The additive inverse of \(r\) is \((-1 \cdot r)\)
\end{fact}

We sure are doing rings here. Man I love rings. Also, I learned about a new \LaTeX package, check this out:
{\huge \fax\ \fax\ \fax} We've even got \Faxmachine\ and \FAX. lol.

So anyway:

\begin{df}
  We say a ring has \textit{multiplicative cancellation} if \(\forall b \neq 0\) and \(a \cdot b = c \cdot b\) then \(a = c\).
\end{df}
\begin{exa}
  This isn't always true. In fact, notice \(\Z/ 6\Z\) does not have multiplicative cancellation.
\end{exa}

\begin{df}
  \(a \in R\) is a \textit{zero divisor} if \(\exists b \in R, b\neq 0\) such that \(a\cdot b = 0\)
\end{df}

\begin{prop}[The worst proposition]
  \(a \in R\) is not a zero divisor iff \(a:R \to R\) by \(a(b) = a \cdot b\) s injective.
\end{prop}


\begin{df}
  An \textit{integral domain} is a non-zero commutative ring without zero divisors.
\end{df}

So which cyclic rings are integral domains? Well notice that when \(n\) is nonprime, \(\Z/n\Z\) is not an integral domain.

Now let's look at some really wacky rings.

\begin{exa}
  Take \(S\) a set, and let \(\mathcal{P}(S)\) be the power set of \(S\). We can make this a ring with \(A \cup B \setminus (A \cap B)\) as addition. Then the additive identity of this ring is the empty set, and a set's inverse is the set complement.

  Then we define multiplication (which will also turn out to be commutative) as \(A \cdot B = A \cap B\), whose identity is the whole set, and whose distributativity is a bit trick to prove.

  \textbf{Hint:} Try drawing a Venn diagram or two.
\end{exa}

Here's another nasty one:

\begin{exa}
  Let \(R\) be a ring, and \(S\) be a set.

  Then \(R^S\) is the function ring of functions \(S \to R\), with \((f + g)(s) = f(s) + g(s)\) and \((f \cdot g)s = f(s)g(s)\). Then the additive identity is : \(f(s) = 0_R\) and multiplicative is \(f(s) = 1_R\). Then this ring has zero divisors in the form of function pairs which take a nonzero value whenever the other takes zero.
\end{exa}

Then something is a (left or right) unit if \(\exists v\) such that \(u \cdot v = 1\) (or visa-versa). A division ring is is a ring where every non zero element is a two sided unit.

\begin{df}
  A \textit{Field} is a non zero commutative ring where every nonzero element is a unit.
\end{df}

\section{Polynomial Rings}

Let \(R\) be a ring. then \(R[x]\) is the set of polynomials in indeterminate \(x\) (consider only polynomials with finitely many terms). Then if \(f = \sum a_ix^i\) and \(g = \sum b_ix^i\), then \(f + g = \sum(a_i + b_i)x^i\), and \(fg = \sum_{k \in \N}\sum_{i + j}a_ib_jx^k\). This is a ring. Have fun checking this you fucks lmao. Now in the same vein, \(R[[x]]\) is the ring of formal power series in \(R\).

\section{Ring Homomorphisms}

\begin{df}
  A \textit{homomorphism of rings} is a map \(\phi\) which is homomorphic on both the addition and the multiplication, and takes the identity of one category to the other.
\end{df}

And naturally an isomorphism is a homomorphism with rings.

Now we can talk about the category or rings, RING. thank god. Then in this category, \(\Z\) is initial. namely, we can construct a homomorphism \(\Z \to R\) which maps \(1 \mapsto 1_R\), \(-1 \mapsto -1_R\), and \(m \mapsto \underbrace{1_R + \ldots + 1_R}_{m\ \mathrm{times}} = m1_R\). Notice that the additive homomorphism falls out trivially. Then in order to get multiplication in general, we need to get that
\begin{align*}
  \phi(m)\phi(n) &= m1_R \cdot n1_R \\
                 &= m \cdot n 1_R 1_R = mn 1_R \\
                 &=\phi(m \cdot n).
\end{align*}

\begin{cor}
  Ring homomorphisms take units to units.
\end{cor}

Now we give a particularly interesting example, despite the fact that rings on their own are boring as shit.

\begin{exa}
  \(\Z[X_1, \ldots, X_n]\), the set of polynomials in some number \(n\) of indeterminants, is the analogue of free groups. And in particular this can be verified using the category \(\mathcal{R}_A\).
\end{exa}
\end{document}
%%% Local Variables:
%%% mode: latex
%%% TeX-master: t
%%% End:

%%% Local Variables:
%%% mode: latex
%%% TeX-master: t
%%% End:
