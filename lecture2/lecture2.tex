\documentclass[12pt, twosided]{article}
\input{sdpreamble}
\graphicspath{{./img/}}

\begin{document}
\noindent \textbf{Math 245} \hfill \textbf{Geniveve Walsh} \\
\textbf{Kyle Dituro} \hfill \textbf{January 22\ndt, 2023}\hrule
\vspace{.2in}

We discuss a little bit more before moving on to ``abstract nonsense''.

Recall that we have the universal property of the product, where something is a product if maps to projections factor through the product. Diagramatically, create a category whose morphisms are the \(\sigma\) such that the below diagram commutes,

\begin{center}
  \begin{tikzcd}
    & Z_1 \arrow[dl, "g_1"] \arrow[d, "\sigma"] \arrow[dr, "f_1"] & \\
    B & Z_2 \arrow[l, "g_2"] \arrow[r, "f_2"] & A
  \end{tikzcd}
\end{center}

and the objects are things of the type
\begin{center}
  \begin{tikzcd}
    & Z \arrow[dl, "g"] \arrow[dr, "f"] & \\
    B & & A
  \end{tikzcd}
\end{center}

Then notice that, in this cattegory, the product

\begin{center}
  \begin{tikzcd}
    & A \times B \arrow[dl, "\pi_B"] \arrow[dr, "\pi_A"] & \\
    B & & A
  \end{tikzcd}
\end{center}

Is final, with morphisms

\begin{center}
  \begin{tikzcd}
    & Z_1 \arrow[dl, "g"] \arrow[d, "\sigma"] \arrow[dr, "f"] & \\
    B & A \times B \arrow[l, "\pi_B"] \arrow[r, "\pi_A"] & A
  \end{tikzcd}
\end{center}

Where \(\sigma(z) = (f(z), g(z))\). Here it is clear that the product map is unique, so this truly is final.

\begin{df}
  We say that a category \(\mathcal{C}\) \textit{has products} if \(\forall A, B \in \mathrm{Obj}(\mathcal{C})\), then \(\mathcal{C}_{A, B}\) has a final object. in SET, final objects are disjoint unions.
\end{df}

\section{Groups}

\begin{df}[Group]
  \(G\) is a set with a closed binary operation
  \begin{align*}
    (\cdot): G \times G \to G \quad \cdot (g, h) \mapsto g \times h
  \end{align*}
  such that
  \begin{enumerate}
  \item \(\cdot\) is associative
  \item \(\exists e_G \in G\) such that \(\forall g \in G\) \(g \cdot e = e \cdot g = g\)
  \item We get inverses: \(\forall g \in G\), there exists \(g\1 \in G\) such that \(g \cdot g\1 = g\1 \cdot g = e_G\).
  \end{enumerate}
\end{df}

\begin{exa}
  The trivial group. Yep.
\end{exa}
\begin{exa}
  \((\Z, +)\), \((\C, +)\), \((\pm 1, \cdot)\). Notice that all of these groups are abelian.
\end{exa}

\begin{exa}
  Matrix groups: \((\mathrm{GL}_n(F),\ \cdot)\), the multiplicative group of invertible \(n \times n\) matrices over a field. Notice that matrix multiplication doesn't commute: a trivial fact.
\end{exa}

\begin{prop}
  Both the identity and \(g\1\) are unique.
\end{prop}
\begin{proof}
  Trivial. Suppose \(\exists e_1, e_2\) both identities for the sake of contradiction. But then their product blah blah blah. And likewise for inverses.
\end{proof}
  \begin{notn}
    By associativity, we can justify that
    \begin{align*}
      g^n = \underbrace{g \cdot g \cdot \ldots \cdot g}_{n \mathrm{times}}
    \end{align*}
  \end{notn}

  As always, a group is called \textit{Abelian} if the binary operation is commutative.

  \begin{exa}
    The Commutator group for a non-abelian group is such that the binary op. is
    \begin{align*}
      [a, b] = aba\1b\1
    \end{align*}

    Likewise the commutator subgroup is a subgroup generated by all commutators of elements from \(G\).
  \end{exa}

  \begin{df}
    The order of an element \(g \in G\) is finite if \(\exists n \in \N\) such that \(g^n = \mathrm{Id}\). The order of the element is the least such \(n\). An element has infinite order otherwise.
  \end{df}

  \begin{lm}
    If \(g^n = e\) for some \(n > 0\), then \(|g|\) divides \(n\).
  \end{lm}
  \begin{proof}
    Trivial exercise in number theory.
  \end{proof}
\end{document}
%%% Local Variables:
%%% mode: latex
%%% TeX-master: t
%%% End:

%%% Local Variables:
%%% mode: latex
%%% TeX-master: t
%%% End:

