\documentclass[12pt, twosided]{article}
\usepackage[letterpaper,bindingoffset=0in,%
            left=1in,right=1in,top=1in,bottom=1in,%
            footskip=.25in]{geometry}

\usepackage{mathtools}
\usepackage{graphicx}

%\usepackage{setspace}
%\setstretch{1.1}

\usepackage{amsmath}
\usepackage{amsfonts}
\usepackage{amsthm}
\usepackage{amssymb}
\usepackage{csquotes}
\usepackage{relsize}

\usepackage{tikz}
\usetikzlibrary{cd}
\usetikzlibrary{fit,shapes.geometric}
\tikzset{%  
    mdot/.style={draw, circle, fill=black},
    mset/.style={draw, ellipse, very thick},
}

\usepackage{hhline}
\usepackage{systeme}
\usepackage{mathrsfs}
\usepackage{hyperref}
\usepackage{mathtools}  
\usepackage{silence}
\usepackage{blkarray}
\usepackage{float}
% \usepackage{framed}
\usepackage{mdframed}
\usepackage{array}
\usepackage{stmaryrd}
\usepackage{extarrows}
\usepackage{caption}
\captionsetup[figure]{labelfont={bf},name={Fig.},labelsep=period}

\makeatletter
\newtheoremstyle{indentedDf}
  {8pt}% space before
  {8pt}% space after
  {\addtolength{\@totalleftmargin}{3.5em}
   \addtolength{\linewidth}{-7em}
   \parshape 1 3.5em \linewidth}% body font
  {}% indent
  {\bfseries}% header font
  {.}% punctuation
  {.5em}% after theorem header
  {}% header specification (empty for default)
\makeatother

\makeatletter
\newtheoremstyle{indentedPlain}
  {8pt}% space before
  {8pt}% space after
  {\itshape
   \addtolength{\@totalleftmargin}{3.5em}
   \addtolength{\linewidth}{-7em}
   \parshape 1 3.5em \linewidth}% body font
  {}% indent
  {\bfseries}% header font
  {.}% punctuation
  {.5em}% after theorem header
  {}% header specification (empty for default)
\makeatother

\theoremstyle{indentedDf}
\newtheorem{df}{Definition}[section]
\newtheorem{exa}[df]{Example}
\newtheorem{ques}[df]{Question}
\newtheorem{exr}[df]{Exercise}
\newtheorem{prb}[df]{Problem}
\newtheorem*{notn}{Notation}
\newtheorem*{note}{Note}

\theoremstyle{indentedPlain}
\newtheorem{thm}[df]{Theorem}
\newtheorem{prop}[df]{Proposition}
\newtheorem{conj}[df]{Conjecture}
\newtheorem{cor}[df]{Corollary}
\newtheorem{lm}[df]{Lemma}
\newtheorem*{fact}{Fact}
\newtheorem*{idea}{Idea}
\newtheorem*{clm}{Claimn}
\newtheorem*{rmk}{Remark}
\usepackage[ruled]{algorithm2e}

\usepackage{ulem}
\makeatletter

\def\lf{\left\lfloor}   
\def\rf{\right\rfloor}
\def\lc{\left\lceil}   
\def\rc{\right\rceil}
\def\st{\text{ s.t. }}
\def\ker{\mathrm{ker}\ }
\def\im{{\mathrm{im}}\ }
\def\1{^{-1}}
\def\2{^2}
\def\3{^3}
\def\tn{^n}
\def\ind{\mathbf{1}}
\def\R{\mathbb{R}}
\def\Q{\mathbb{Q}}
\def\Z{\mathbb{Z}}
\def\C{\mathbb{C}}
\def\I{\mathbb{I}}
\def\N{\mathbb{N}}
\def\F{\mathbb{F}}
\def\A{\mathbb{A}}
\def\Li{\text{Li}}
\def\th{^\text{th}}
\def\sp{\text{Sp}}
\def\opn{\left\{}
\def\cls{\right\}}
\def\Aut{\text{Aut}}
\def\PG{\text{PG}}
\def\GL{\text{GL}}
\def\PGL{\text{PGL}}
\def\Cov{\text{Cov}}
\def\Pack{\text{Pack}}
\def\PgamL{\text{P}\Gamma\text{L}}
\def\gamL{\Gamma\text{L}}
\def\cl{\text{cl}}
\def\stbar{\ \middle\vert\ }
\def\partdone{\hphantom{1} \hfill \(\triangle\)}
\def\s0{_0}
\def\s1{_1}
\def\s2{_2}
\def\id{\mathrm{id}}
\def\topn{\text{ open}}
\def\Bd{\text{Bd }}
\def\nope{\(\longrightarrow\!\!\longleftarrow\)}
\def\stt{\(^{\text{st}}\)}
\def\tht{\(^{\text{th}}\)}
\def\ndt{\(^{\text{nd}}\)}
\def\t{^{T}}
\def\c{^c}
\newcommand\norm[1]{\left\lVert#1\right\rVert}
\renewcommand{\P}{\mathbb{P}}
\newcommand{\leg}[2]{\left( \frac{#1}{#2} \right)}

\renewcommand*\env@matrix[1][*\c@MaxMatrixCols c]{%
   \hskip -\arraycolsep
   \let\@ifnextchar\new@ifnextchar
   \array{#1}}
\makeatother

% These two lines suppress the warning generated 
% by amsmath for overwriting the choose command  
% because it's annoying. This probably has unint-
% ended ramifications somewhere else, but I'm too
% lazy to actually figure that out, so we'll cro-
% ss that bridge when we come to it lol.
\renewcommand{\choose}[2]{\left( {#1 \atop #2} \right)}
\WarningFilter{amsmath}{Foreign command} 

\renewcommand{\mod}[1]{\ (\mathrm{mod}\ #1)}
\renewcommand{\vec}[1]{\mathbf{#1}}

\let\oldprime\prime
\def\prime{^\oldprime}

\usepackage{float}
\restylefloat{figure}

\usepackage{cleveref}
\Crefname{thm}{Theorem}{Theorems}

% Comment commands for co-authors
\newcommand{\kmd}[1]{{\color{purple} #1}}

\newcolumntype{L}{>{$}l<{$}}
% Bib matter
\let\oldepsilon\epsilon
\def\epsilon{\varepsilon}

\let\oldphi\phi
\def\phi{\varphi}

% New definition of square root:
% it renames \sqrt as \oldsqrt
\let\oldsqrt\sqrt
% it defines the new \sqrt in terms of the old one
\def\sqrt{\mathpalette\DHLhksqrt}
\def\DHLhksqrt#1#2{%
\setbox0=\hbox{$#1\oldsqrt{#2\,}$}\dimen0=\ht0
\advance\dimen0-0.2\ht0
\setbox2=\hbox{\vrule height\ht0 depth -\dimen0}%
{\box0\lower0.4pt\box2}}
%%% Local Variables:
%%% mode: plain-tex
%%% TeX-master: t
%%% End:

\graphicspath{{./img/}}

\begin{document}
\noindent \textbf{Math 245} \hfill \textbf{Genevieve Walsh} \\
\textbf{Kyle Dituro} \hfill \textbf{January 18, 2023}\hrule
\vspace{.2in}

\section{Categories}
The unifying framework for our study of algebra will be that of categories. So far as test material goes, categories will likely not be tested

Roughly, a \textit{category} is a class of objects with some notion of morphisms between them.

\begin{exa}
  Sets with functions of sets, groups with group homomorphisms.
\end{exa}

\begin{df}
  A  \textit{Category} will be denoted \(\mathcal{C}\)
  \begin{itemize}
  \item \(\mathrm{Obj}(\mathcal{C})\) is the collection of objects,
  \item For every pair \((A, B)\), \(A, B \in \mathcal{C}\) (via abuse of notation)
  \item A set \(\mathrm{Hom}_\mathcal{C}(A, B)\).
  \end{itemize}

  Satisfying
  \begin{itemize}
  \item \(\forall A \in \mathrm{Obj}(\mathcal{C})\), \(\exists 1_A \in \mathrm{Hom}_{\mathcal{C}}(A, A)\)
  \item \(\forall f, g\), \(f \in \mathrm{Hom}_\mathcal{C}(A,B)\), \(g \in \mathrm{Hom}_\mathcal{C}(B, C)\), \(\exists ggf \in \mathrm{Hom}_\mathcal{C}(A,C)\).
  \item \(f \in \mathrm{Hom}(A, B)\), \(g \in \mathrm{Hom}(B,C)\), \(h \in \mathrm{Hom}(C, D)\), then \((hg)f = h(gf)\)
  \item Composition plays well with the identity

    \(\forall f \in \mathrm{Hom}(A, B)\), \(f1_A = f, 1_Bf = f\)
  \item And ``non-craziness'' \(\mathrm{Hom}_\mathcal{C}(A,B) \cap \mathrm{Hom}_\mathcal{C}(E, F) = \emptyset\) unless \(A = E\), \(B = F\).
  \end{itemize}
\end{df}

As a matter of vocabulary, we will say \(f \in \mathrm{Hom}_\mathcal{C}(A, A)\) is called an \textit{endomorphism} and the collection of all endomorphisms will be denoted \(\mathrm{End}_\mathcal{C}(A)\), and moreover for \(f \in \mathrm{Hom}_\mathcal{C}(A,B)\)  we will often write \(f: A \to B\) as usual as a slight abuse of notation.

\begin{exa}
  SET is a category where \(\mathrm{Obj}(\mathrm{SET})\) is sets, and \(\mathrm{Hom}_\mathrm{SET}(A, B) = B^A\).
\end{exa}
\begin{exa}
  Also real vector spaces and linear maps is another category
\end{exa}
Now suppose we have \(\mathcal{C}\), \(A \in \mathrm{Obj}(\mathcal{C})\). We define a new category \(\mathcal{C}_A\).

\begin{df}
  For a category \(\mathcal{C}\), define \(\mathcal{C}_A\) as
  \begin{itemize}
  \item \(\mathrm{Obj}(\mathcal{C}_A) = Z\), all \(f \in \mathrm{Hom}(Z, A)\) for \(Z \mathrm{Obj}(\mathcal{C})\)
  \item \(\mathrm{Hom}_{\mathcal{C}_A} = \opn \sigma \in \mathrm{Hom}_\mathcal{C}(Z, X) \stbar f = g\sigma\cls\).
  \end{itemize}
\end{df}
It is fairly routine to check that the diagram

\begin{center}
  \begin{tikzcd}
    Z \arrow[dr, "f_1"] \arrow[r, "\sigma"]& X  \arrow[d, "f_2"] \arrow[r, "\beta"]& Y \arrow[dl, "f_3"]\\
    & A & 
  \end{tikzcd}
\end{center}

commutes, which checks composition.

As usual, we will call \(f \in \mathrm{Hom}(A, B)\) an \textit{isomorphism} of categories if it has an inverse.

\begin{prop}
  An inverse of an isomorphism is unique.
\end{prop}
\begin{proof}
  \begin{align*}
    f \in \mathrm{Hom}_\mathcal{C}(A, B) \quad g_1, g_2 \in \mathrm{Hom}_\mathcal{C}(B, A). \\
    g_1 = g_11_B = g_1(fg_2) = (g_1f)g_2 = 1_Bg_2 = g_2
  \end{align*}
\end{proof}

Some other basic properties of isomorphisms:

\begin{itemize}
\item the identity is an isomorphism
\item \(f\) is an isomorphism iff \(f\1\) is as well
\item if \(f\) and \(g\) are isomorphisms with compatible domain and codomains, then \(gf\) is also an isomorphism.
\end{itemize}

Now we can discuss automorphisms. We denote the collection of all automorphisms of an object \(A\) as \(\mathrm{Aut}_\mathcal{C}(A) \subseteq \mathrm{End}_\mathcal{C}(A)\), and we have that
\begin{itemize}
\item Identity is just \(1_A\)
\item associative multiplication
\item and every element has an inverse.
\end{itemize}
Therefore, this is a group.

\subsection{Universal Properties}
Suppose that we have some category \(\mathcal{C}\). Then say that \(I \in \mathrm{Obj}(C)\) is \textit{initial} in \(\mathcal{C}\) if, for every object \(K \in \mathrm{Obj}(C)\), \(\exists ! f: I \to K\). Likewise, we define an object \(F\) to be \textit{final} in \(\mathcal{C}\) us \(\forall K \in \mathrm{Obj}(C)\), \(\exists ! f \in \mathrm{Hom}(K, F), f: K \to F\).

\begin{exa}
  All sets with one element are final objects in SET, and the empty set is the only initial objects.

  Note that here all initial objects are isomorphic, and likewise with all final objects. This is true in general, and is trivial to prove.
\end{exa}

Call an object \textit{terminal} if it is either an initial or a final object.

\begin{df}
  A construction is a \textit{universal property} if it is a terminal object in some category. Often times the category is unsaid.
\end{df}

\subsection{Quotients}

Denote quotients by an equivalence relation as normal. Then the quotient \(A / \sim\) is universal with respect to the property of mapping \(A\) to some set where equivalent elements have the same image. Obviously, there is a natural map \(A \xlongrightarrow{\pi} A / \sim\) mapping elements of \(A\) to it's equivalence class.

Lisewise \(\mathcal{C}_\sim\) is the category of objects \(f \in \mathrm{Hom}_\mathrm{SET}(A, Z)\) where \(a \sim a\prime \implies f(a) = f(a\prime), \forall Z \in \mathrm{Obj}(\mathrm{SET})\)  with morphisms \(\sigma \in \mathrm{Hom}(Z_1, Z_2)\) such that \(\sigma f_2 = f_1\). Then the claim is that \(\pi\) as defined before is initial in this category.

See that \(\exists !\) morphism \(\pi \to f\),

\begin{center}
  \begin{tikzcd}
    & A \arrow[dl, "\pi"] \arrow[dr, "f"]& \\
    A / \sim  \arrow[rr, "\sigma"] & & Z
  \end{tikzcd}
\end{center}
\end{document}
%%% Local Variables:
%%% mode: latex
%%% TeX-master: t
%%% End:

%%% Local Variables:
%%% mode: latex
%%% TeX-master: t
%%% End:
